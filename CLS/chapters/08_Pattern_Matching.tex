%!TEX TS-program = xelatex
%!TEX encoding = UTF-8 Unicode

\chapter{Pattern Matching}
\label{sec:pattern-matching}






\section{Patterns}
\label{sec:patterns}

\syntax\begin{lstlisting}
Pattern        ::= Pattern1 {'|' Pattern1}
Pattern1       ::= ['implicit'] var_id ':' Type_Pat
                 | '_' ':' Type_Pat
                 | Pattern2
Pattern2       ::= ['implicit'] var_id ['@' Simple_Pattern]
                 | Simple_Pattern
Simple_Pattern ::= ['*'] '_'
                 | ['*'] var_id
                 | literal
                 | Stable_Id
                 | Stable_Id '(' [Patterns] ')' 
                 | '(' [Patterns] ')'
                 | Pattern '&' Pattern
                 | Pattern '~>' Pattern
                 | Pattern '<~' Pattern
                 | Pattern ('..' | '...') Pattern
Patterns       ::= Pattern {',' Pattern}
\end{lstlisting}






\subsection{Variable Patterns}
\label{sec:variable-patterns}

\syntax\begin{lstlisting}
Simple_Pattern ::= '_'
                 | var_id
\end{lstlisting}

A variable pattern $x$ is a simple identifier which starts with a lower case letter. It matches any value and binds the variable name to that value. The type of $x$ is the expected type of the pattern as given from the outside. A special case is the wild-card pattern ``\lstinline!_!'', which is treated as if it was a fresh variable on each occurence, and which does not bind itself to the value, i.e., it is alone equivalent to the \code{else} clause of \code{When_Clauses}. 






\subsection{Typed Patterns}
\label{sec:typed-patterns}

\syntax\begin{lstlisting}
Simple_Pattern ::= '_' ':' Type_Pat
                 | var_id ':' Type_Pat
Type_Pat       ::= Type
\end{lstlisting}

A typed pattern ~\lstinline!$x$: $T$!~ consists of a pattern variable $x$ and a type pattern $T$. The type of $x$ is the type $T$, where each type variable and wildcard is replaced by a fresh, unknown type. This pattern matches any value matched by the type pattern $T$ (\sref{sec:type-patterns}), and it binds the variable name to that value (unless the variable name is ``\lstinline!_!''). 






\subsection{Pattern Binders}
\label{sec:pattern-binders}

\syntax\begin{lstlisting}
Pattern2 ::= var_id '@' Simple_Pattern
\end{lstlisting}

A pattern binder ~\lstinline!$x$ @ $p$!~ consists of a pattern variable $x$ and a pattern $p$. The type of the variable $x$ is the type $T$ resulting from the pattern $p$. This pattern matches any value $v$ matched by the pattern $p$, provided the type of $v$ is also an instance of $T$, and it binds the variable name to that value. 

\example In the following example, \code{person} binds to the whole \code{Person} object. 
\begin{lstlisting}
def f (someone: Person) := match someone
  when person @ Person('John Galt', _, _) then $\ldots$
end match
\end{lstlisting}







\subsection{Literal Patterns}
\label{sec:literal-patterns}

\syntax\begin{lstlisting}
Simple_Pattern ::= literal
\end{lstlisting}

A literal pattern $L$ matches any value that is equal (in terms of ~\lstinline!=!) to the literal $L$. The type of $L$ must conform to the expected type of the pattern. Literal kinds that are considered legal with this pattern are: string literals, number literals, boolean literals and the \code{nil} value. 





\subsection{Stable Identifier Patterns}
\label{sec:stable-identifier-patterns}

\syntax\begin{lstlisting}
Simple_Pattern ::= Stable_Id
\end{lstlisting}

A stable identifier pattern is a stable identifier $r$ (\sref{sec:type-paths}). The type of $r$ must conform to the expected type of the pattern. The pattern matches any value $v$, such that ~\lstinline!$r$ = $v$!.

To resolve the syntactic overlap with a variable pattern (\sref{sec:variable-patterns}), a stable identifier pattern may not be a simple name starting with a lower case letter. However, it is possible to enclose such a variable or method name in backquotes, then it is treated as a stable identifier pattern. 

\example Consider the following function definition:
\begin{lstlisting}
def f (x: Integer, y: Integer) := match x
  when y then $\ldots$
end match
\end{lstlisting}
Here, \code{y} is a variable pattern, which matches any value, namely here it would bind simply to \code{x}. If we wanted to turn the pattern into a stable identifier pattern, this can be achieved as follows:
\begin{lstlisting}
def f (x: Integer, y: Integer) := match x
  when `y` then $\ldots$
end match
\end{lstlisting}
Now, the pattern matches the \code{y} parameter of the enclosing function \code{f}. That is, the match succeeds only if the \code{x} argument and the \code{y} argument of \code{f} are equal. 






\subsection{Constructor Patterns}
\label{sec:constructor-patterns}

\syntax\begin{lstlisting}
Simple_Pattern ::= Stable_Id '(' [Patterns] ')'
\end{lstlisting}

A constructor pattern is of the form ~\lstinline!$c$($p_1 \commadots p_n$)!, for $n \geq 0$. It consists of a stable identifier $c$, followed by element patterns $p_1 \commadots p_n$. The constructor $c$ is a simple or qualified name which denotes a case class (\sref{sec:case-classes}). If the case class is monomorphic, then it must conform to the expected type of the pattern, and the formal parameter types of $c$'s primary constructor (\sref{sec:constructor-destructor-def}) are taken as the expected types of the element patterns $p_1 \commadots p_n$. If the case class is polymorphic, then its type parameters are instantiated so that the instantiation of $c$ conforms to the expected type of the pattern, unless the type arguments are already given. These types of the formal parameter types of $c$'s primary constructor are then taken as the expected types of the component patterns $p_1 \commadots p_n$. The pattern matches all objects created from constructor invocations ~\lstinline!$c$($p_1 \commadots p_n$)!, where each element pattern $p_i$ matches the corresponding value $v_i$. Any extra parameter sections of $c$'s primary constructor do not affect this behavior. 

A special case arises when $c$'s formal parameter types contain a repeated parameter. This is further discussed in (\sref{sec:pattern-sequences}).






\subsection{Tuple Patterns}
\label{sec:tuple-patterns}

\syntax\begin{lstlisting}
Simple_Pattern ::= '(' [Patterns] ')'
\end{lstlisting}

A tuple pattern ~\lstinline!($p_1 \commadots p_n$)!~ is an alias for the constructor pattern ~\lstinline!Tuple_$n$($p_1 \commadots p_n$)!, where $n \geq 2$. The empty tuple ~\lstinline!()!~ is the unique value of type \code{Unit}. 





\subsection{Extractor Patterns}
\label{sec:extractor-patterns}

\syntax\begin{lstlisting}
Simple_Pattern ::= Stable_Id '(' [Patterns] ')'
\end{lstlisting}

An extractor pattern ~\lstinline!$x$($p_1 \commadots p_n$)!, where $n \geq 0$, is of the same syntactic form as a constructor pattern. However, instead of a case class, the stable identifier $x$ denotes an object which has a member method named \code{unapply} or \code{unapply_sequence} that matches the pattern. 

An \code{unapply} method in an object $x$ {\em matches} the pattern ~\lstinline!$x$($p_1 \commadots p_n$)!~ if it takes exactly one argument and one of the following applies: 
\begin{itemize}
\item[]
$n = 0$ and \code{unapply}'s result type is \code{Boolean}. In this case the extractor pattern matches all values $v$ for which ~\lstinline!$x$.unapply($v$)!~ returns \code{yes}. 
\item[]
$n = 1$ and \code{unapply}'s result type is ~\lstinline!Option[$T$]!, for some type $T$. In this case, the only argument pattern $p_1$ is typed in turn with expected type $T$. The extractor pattern matches then all values $v$ for which ~\lstinline!$x$.unapply($v$)!~ returns a value of form ~\lstinline!Some($v_1$)!, and $p_1$ matches $v_1$. 
\item[]
$n > 1$ and \code{unapply}'s result type is ~\lstinline!Option[($T_1 \commadots T_n$)]!, for some types $T_1 \commadots T_n$. In this case, the argument patterns $p_1 \commadots p_n$ are typed in turn with expected types $T_1 \commadots T_n$. The extractor pattern matches then all values $v$ for which ~\lstinline!$x$.unapply($v$)!~ returns a value of form ~\lstinline!Some(($v_1 \commadots v_n$))!, and each pattern $p_i$ matches the corresponding value $v_i$.
\end{itemize}

An \code{unapply_sequence} method in an object $x$ matches the pattern ~\lstinline!$x$($p_1 \commadots p_n$)!, if it takes exactly one argument and its result type is of the form ~\lstinline!Option[$S$]!, where $S$ is a subtype of ~\lstinline!Sequence[$T$]!~ for some element type $T$. This case is further discussed in (\sref{sec:pattern-sequences}).





\subsection{Pattern Sequences}
\label{sec:pattern-sequences}

\syntax\begin{lstlisting}
Simple_Pattern ::= Stable_Id '(' [Patterns ','] 
                   '*' (var_id | '_')
                   [',' Patterns] ')'
\end{lstlisting}

A pattern sequence $p_1 \commadots p_n$ appears in two contexts. First, in a constructor pattern ~\lstinline!$c$($q_1 \commadots q_a, p_1 \commadots p_n, r_1 \commadots r_b$)!, where $c$ is a case class, which has $a+1+b$ primary constructor parameters, with a repeated parameter (\sref{sec:repeated-parameters}) of type ~\lstinline!*$S$! in the middle. Second, in an extractor pattern ~\lstinline!$x$($p_1 \commadots p_n$)!, if the extractor object $x$ has an \code{unapply_sequence} method with a result type conforming to ~\lstinline!Sequence[$T$]!, but does not have an \code{unapply} method that matches $p_1 \commadots p_n$. The expected type for the pattern sequence is in each case the type $S$. 

% TBD: elaborate on how the element patterns match






\subsection{Conjunction Patterns}
\label{sec:conjunction-patterns}

\syntax\begin{lstlisting}
Simple_Pattern ::= Pattern '&' Pattern
\end{lstlisting}

A conjunction pattern ({\em {\normalfont ``and''} pattern}) matches only if both patterns match. Moreover, only the first pattern in a sequence of conjunction patterns may bind variable names, but variables from the other patterns have their scope extended to the following patterns. This behavior is unlike in pattern alternatives, which aren't allowed to bind variable names at all, due to the fact that each alternative may match a completely different structure, whereas a conjunction pattern matches on the same structure. 





\subsection{List Patterns}
\label{sec:list-patterns}

\syntax\begin{lstlisting}
Simple_Pattern ::= Pattern '~>' Pattern
                 | Pattern '<~' Pattern
\end{lstlisting}

A list pattern is a pattern specialized to match elements of list instances. A list pattern of the form ~\lstinline!$p_1$ ~> $p_2$!~ matches a list, where $p_1$ matches the head element of the list and $p_2$ matches the remainder of the list, ending with the tail element. A list pattern of the form ~\lstinline!$p_1$ <~ $p_2$!~ matches a list, where $p_1$ matches the tail element of the list and $p_2$ matches the remained of the list, starting with the head element. If the list has one element, then the matched remainder is \code{nil}. 





\subsection{Range Patterns}
\label{sec:range-patterns}

\syntax\begin{lstlisting}
Simple_Pattern ::= Pattern ('..' | '...') Pattern
\end{lstlisting}

A range pattern is a pattern specialized to match scalar values with ordering.






\subsection{Pattern Alternatives}
\label{sec:pattern-alternatives}

\syntax\begin{lstlisting}
Pattern ::= Pattern {'|' Pattern}
\end{lstlisting}

A pattern alternative ~\lstinline!$p_1$ | $\ldots$ | $p_n$!, where $n \geq 2$, consists of a number of alternative patterns $p_i$. All alternative patterns are type checked with the expected type of the pattern. They may not bind variable names other than wildcards (which are discarded). The alternative pattern matches a value $v$ if at least one of its alternatives matches $v$. Consequently, if a first such match is successful, the remaining patterns are not tested.

{\em Non-normatively:} Usually, each pattern alternative is a literal pattern, and therefore binding a variable name makes no sense, since the value that the pattern matching expression matches against is the already bound variable. Still, if needed, the pattern alternatives may be used together with a pattern binder (\sref{sec:pattern-binders}), which can be useful when the structure that is being matched contains union types. 





\subsection{Regular Expression Patterns}
\label{sec:regexp-patterns}

\syntax\begin{lstlisting}
Simple_Pattern ::= regexp_literal
\end{lstlisting}

A regular expression pattern $p$ ({\em regexp pattern}) is a variant of literal pattern, designed to match \code{String_Like} values. Literally, the pattern $p $ matches a value $v$, if $v$ is of a type that conforms to \code{String_Like} and its contents match the regular expression. Moreover, sub-patterns bind to variable names of a name of the form ~\lstinline!match_$n$!, where $n$ is either the position of the sub-pattern (unless the sub-pattern is explicitly not captured), or the name of a named sub-pattern. 

Regular expression patterns are not designed to match against algebraic data structures. 

\subsection{Irrefutable Patterns}
\label{sec:irrefutable-patterns}

A patter $p$ is {\em irrefutable} for a type $T$, if one of the following applies: 
\begin{enumerate}
\item $p$ is a variable pattern (\sref{sec:variable-patterns}),
\item $p$ is a typed pattern ~\lstinline!$x$: $T'$! (\sref{sec:typed-patterns}), and ~\lstinline!$T$ <: $T'$!,
\item $p$ is a constructor pattern ~\lstinline!$c$($p_1 \ldots p_n$)! (\sref{sec:constructor-patterns}), the type $T$ is an instance of a class $c$, the primary constructor (\sref{sec:class-definitions}) of type $T$ has argument types $T_1 \commadots T_n$, and each $p_i$ is irrefutable for type $T_i$. 
\end{enumerate}






\section{Type Patterns}
\label{sec:type-patterns}

\syntax\begin{lstlisting}
Type_Pat ::= Type
\end{lstlisting}

Type patterns consist of types, type variables and wildcards. A type pattern $T$ is of one of the following forms: 
\begin{itemize}
\item[]
A reference to a class $C$, ~\lstinline!$p$.$C$!, ~\lstinline!$p$.type!~ or ~\lstinline!$T$#$C$!. This type pattern matches any non-\code{nil} instance of the given class (therefore, it does match the empty tuple ~\lstinline!()!~ with type \code{Unit}). Note that the prefix of the class, if it is given, is irrelevant for determining class instances, unlike in Scala. 

The bottom type \code{Nothing} (with singleton instance \code{nil}) is the only type pattern that matches \code{nil} (only), but its preferable to match against ~\lstinline!Option[$T$]! with implicit conversion of \code{nil} to object \code{None}. 

\item[]
A singleton type ~\lstinline!$p$.type!. This type pattern matches only the value denoted by the path $p$ (only the single value denoted by the path $p$, since \code{nil} is not matched). 

\item[]
A parameterized type pattern ~\lstinline!$T$[$a_1 \ldots a_n$][<$t_1 \ldots t_n$>]!, where the $a_i$ are type variable patterns or wildcards ``\lstinline!_!'' and $u_i$ are unit of measure kinds. This type pattern matches all values which match $T$ for some arbitrary instantiation of the type variables and wildcards.  

\item[]
A compound type pattern ~\lstinline!$T_1$ with $\ldots$ with $T_n$!, where each $T_i$ is a type pattern. This type pattern matches all values that are matched by each of the type patterns $T_i$, and in this sense it is equivalent to the pattern ~\lstinline!$T_1$ & $\ldots$ & $T_n$!. 
\end{itemize}

Types are not subject to any type erasure, so it is basically safe to use any other type as type pattern (e.g. constrained types -- \sref{sec:constrained-types}), unlike in Scala. 

A {\em type variable pattern} is a simple identifier which starts with a lower case letter. 






\section{Pattern Matching Expressions}

\syntax\begin{lstlisting}
Match_Expr     ::= Pat_Match_Expr | Case_Expr
Pat_Match_Expr ::= 'match' Simple_Expr1 Match_Body
Match_Body     ::= semi When_Clauses 'end' ['match']
                 | '{' When_Clauses '}'
When_Clauses   ::= When_Clause {['next'] semi When_Clause} 
                   [['next'] 'else' Block]
When_Clause    ::= 'when' Pattern [Guard] ('then' | semi) Block
\end{lstlisting}

A pattern matching expression 
\begin{lstlisting}
match $e$ { when $p_1$ then $b_1\ \ldots$ when $p_n$ then $b_n$ else $b_{n+1}$ }
\end{lstlisting}
consists of a selector expression $e$ and a number $n > 0$ of cases. Each case consists of a (possibly guarded) pattern $p_i$, a block $b_i$ and optionally the default block $b_{n+1}$, if none of the patterns matched. Each $p_i$ might be complemented by a guard \code{if e} or \code{unless e}, where $e$ is a guarding expression, that is typed as \code{Boolean}. The scope of the pattern variables in $p_i$ comprises the pattern's guard and the corresponding block $b_i$. If the following when clause ~\lstinline!when $p_{i+1}$ then $b_{i+1}$! is preceded by the keyword \code{next}, then the pattern variables in $p_i$ do not comprise the block $b_{i+1}$ and neither the pattern $p_{i+1}$. 

Let $T$ be the type of the selector expression $e$. Every pattern $e \in \{ p_1 \ldots p_n \}$ is typed with $T$ as its expected type. 

The expected type of every block $b_i$ is the expected type of the whole pattern matching expression. The type of the pattern matching expression is then the weak least upper bound (\sref{sec:weak-conformance}) of the types of all blocks $b_i$. 

When applying a pattern matching expression to a selector value, patterns are tried in given order, until one is found that matches the selector value. Say this \code{when} clause is ~\lstinline!when $p_i$ then $b_i$!. The result of the whole expression is then the result of evaluating $b_i$, where all pattern variables of $p_i$ are bound to the corresponding parts of the selector value. If no matching pattern is found, a \code{No_Match_Error} is raised. 

The pattern in a \code{when} clause may also be followed by a guard suffix ~\lstinline!if $e$! with a boolean expression $e$. The guard expression is evaluated if the preceding pattern in the case matches. If the guard expression evaluates to \code{yes}, the pattern match succeeds as normal. If the guard expression evaluates to \code{yes}, the pattern in the case is considered not to match and the search for a matching pattern continues. 

The pattern in a case may also be followed by a guard suffix ~\lstinline!unless $e$! with a boolean expression $e$. The guard expression is evaluated as if it was ~\lstinline@if !$e$@. 

In the interest of efficiency the evaluation of a pattern matching expression may try patterns in some other order than the textual sequence, even parallelized (indeed, compiler would not decide this on its own -- it has to be specified with an annotation or a pragma (\sref{sec:annotations}) applied to the pattern matching expression). This might affect evaluation through side effects in guards. However, it is guaranteed that a guard expression is evaluated only if the pattern it guards matches.

If the selector of a pattern match is an instance of a \code{sealed} class (\sref{sec:modifiers}), the compilation of the pattern matching expression can emit warnings, which diagnose that a given set of patterns is not exhaustive, i.e. there is a possibility of a \code{No_Match_Error} being raised at runtime. 






\section{Pattern Matching Anonymous Functions}
\label{sec:pattern-matching-anon-fun}

\syntax\begin{lstlisting}
Block_Expr ::= '{' When_Clauses '}'
\end{lstlisting}

An anonymous function can be defined by a sequence of cases
\begin{lstlisting}
{ when $p_1$ then $b_1$ $\ldots$ when $p_k$ then $b_k$ else $b_{k+1}$ }
\end{lstlisting}
which appears as an expression without a prior \code{match}. The expression is expected to be of a block type, unless it is expected to be a function type, in that case it is converted to ~\lstinline!Function_$k$[$S_1 \commadots S_k$, $T$]!~ automatically for $k \geq 1$, and for $k = 0$ the expression does not match the expected function type in function applications (\sref{sec:function-applications}).\footnote{This is due to Coral not knowing the expected types in advance, but this anonymous function expression is able to match any non-empty arguments list, which is simply passed into the implicit pattern matching expression.} 

The expression is taken to be equivalent to the anonymous function:
\begin{lstlisting}
($x_1$: $S_1 \commadots x_k$: $S_k$) -> {
  match ($x_1 \commadots x_k$) {
    when $p_1$ then $b_1$
    $\ldots$
    when $p_k$ then $b_k$
    else $b_{k+1}$
  }
}
\end{lstlisting}

Here, each $x_i$ is a fresh name. As was shown in (\sref{sec:anonymous-functions}), this anonymous function is in turn equivalent to the following instance creation expression, where $T$ is the weak least upper bound of the types of all $b_i$. 

\begin{lstlisting}
(Function_$k$[$S_1 \commadots S_k$, $T$] with {
  def apply ($x_1$: $S_1 \commadots x_k$: $S_k$): $T$ := match ($x_1 \commadots x_k$)
    when $p_1$ then $b_1$
    $\ldots$
    when $p_k$ then $b_k$
    else $b_{k+1}$
  end match
}).new
\end{lstlisting}

\example Here is a method which uses a fold-left operation ~\lstinline!/:!~ to compute the scalar product of two vectors:
\begin{lstlisting}
def scalar_product (xs: List[Double], ys: List[Double]) := 
  (0.0 /: xs.zip(ys)) {
    when (a, (b, c)) then a + b * c
  }
\end{lstlisting}
The when clauses in this code are equivalent to the following anonymous function:
\begin{lstlisting}
(x, y) -> { 
  match (x, y) {
    when (a, (b, c)) then a + b * c
  }
}
\end{lstlisting}
Note that the fold-left operation ~\lstinline!/:!~ is an operator ending in a colon ``\lstinline!:!'', and therefore right-associative, and therefore the expression is interpreted as specified in (\sref{sec:infix-operations}) for right-associative operations: ~\lstinline[mathescape=false]!{ val x$ := 0.0; xs.zip(ys).`/:`(x) }!. 











