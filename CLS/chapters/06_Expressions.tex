%!TEX TS-program = xelatex
%!TEX encoding = UTF-8 Unicode

\chapter{Expressions}

% TBD: update this chapter and the syntax gradually as the CLS evolves

\syntax\begin{lstlisting}
Expr        ::= Cond_Expr
              | Loop_Expr
              | Rescue_Expr
              | Raise_Expr
              | Throw_Expr
              | Catch_Expr
              | Return_Expr
              | Assign_Expr
              | Update_Expr
              | Yield_Expr
              | Prefix_Expr
              | Simple_Expr
              | Match_Expr
              | Binding
              | Annot_Expr
              | Cast_Expr
Simple_Expr ::= '(' (Class_Template | Template_Body) ')'
              | Block_Expr
              | Simple_Expr1 ['_']
\end{lstlisting}

Expressions are composed of various keywords, operators and operands. Expression forms are discussed subsequently. 







\section{Expression Typing}

The typing of expressions is often relative to some {\em expected type} (which might be undefined). When we write ``expression $e$ is expected to conform to type $T$'', we mean:
\begin{enumerate}
\item The expected type of $e$ is $T$.
\item The type of expression $e$ must conform to $T$. 
\end{enumerate}

Usually, the type of the expression is defined by the last element of an execution branch, as discussed subsequently with each expression kind. 

What we call ``statement'', in context of Coral is in fact yet another kind of an expression, and those expressions themselves always have a type and a value. 





\section{Literals}

\syntax\begin{lstlisting}
Simple_Expr1 ::= literal
\end{lstlisting}

Typing of literals is as described in (\sref{sec:literals}); their evaluation is immediate, including non-scalar literals (collection literals). 






\subsection{The Nil Value}

\syntax\begin{lstlisting}
Simple_Expr1 ::= 'nil'
\end{lstlisting}

The \code{nil} value is of type \code{Nothing}, and is thus compatible with every type that is nullable (\sref{sec:nullability}), either preferably or explicitly.

The \code{nil} represents a ``no object'', and is itself represented by an object. This object overrides methods in \code{Object} as follows: 
\begin{itemize}
\item 
\lstinline!equals($x$)! and \lstinline!=($x$)! return \code{yes} if the argument $x$ is also the \code{nil} object. 

\item 
\lstinline@!=($x$)@ return \code{yes} if the argument $x$ is not the \code{nil} object.

\item
\lstinline[mathescape=false]!as_instance_of:[$T]()! returns always \code{nil}. 

\item
\lstinline!hash_code()! returns \code{0}. 
\end{itemize}

A reference to any other member of the \code{nil} object causes \code{Method_Not_Found_Error} or \code{Member_Not_Found_Error} to be raised, unless the member in fact exists.\footnote{It is even possible to use a refinement to actually implement some methods of \code{nil} locally (preferred approach), or globally implement those methods (discouraged, causes warnings).} 






\section{Designators}

\syntax\begin{lstlisting}
Simple_Expr1 ::= Path
               | Simple_Expr '.' importable_id
\end{lstlisting}

A designator refers to a named term. It can be a {\em simple name} or a {\em selection}.






\section{Self, This \& Super}
\label{sec:self-this-super}

\syntax\begin{lstlisting}
Simple_Expr1 ::= [Container_Path '.'] 'self'
                 ['.' (constant_id | variable_id | function_id)]
               | [Container_Path '.'] 'this'
                 ['.' (constant_id | variable_id | function_id)]
               | [Container_Path '.'] 'super' 
                 [Class_Qualifier] 
                 ['.' (constant_id | variable_id | function_id)]
\end{lstlisting}

The expression \code{self} stands always for the current instance in the context and function resolution searches in the actual class of the instance. 

The expression \code{this} is the same as \code{self}, except that function resolution searches from the class that this expression appears in, possibly skipping overrides. 

% TBD: expand the description






\section{Use Expressions}
\label{sec:use-expressions}








\section{Function Applications}
\label{sec:function-applications}

\subsection{Named and Optional Arguments}
\label{sec:named-optional-arguments}

\subsection{By-Name Arguments}
\label{sec:by-name-arguments}

\subsection{Input \& Output Arguments}
\label{sec:io-arguments}

\subsection{Function Compositions \& Pipelines}

\section{Method Values}

\section{Type Applications}

\section{Tuples}

\section{Instance Creation Expressions}

\section{Blocks}

\subsection{Local Variable Closure}
\label{sec:local-variable-closure}

\section{Prefix \& Infix Operations}

\subsection{Prefix Operations}

\subsection{Infix Operations}

\subsection{Assignment Operators}

\section{Typed Expressions}

\section{Annotated Expressions}
\label{sec:annotated-exprs}

\section{Assignments}

\section{Conditional Expressions}

\section{Loop Expressions}

\subsection{Classic For Expressions}

\subsection{Iterable For Expressions}

\subsection{Basic Loop Expressions}

\subsection{While \& Until Loop Expressions}

\subsection{Conditions in Loop Expressions}

\section{Collection Comprehensions}

\section{Return Expressions}

\subsection{Implicit Return Expressions}

\subsection{Explicit Return Expressions}

\subsection{Structured Return Expressions}

\section{Raise Expressions}

\section{Rescue \& Ensure Expressions}


\section{Throw \& Catch Expressions}

\section{Anonymous Functions}

\section{Conversions}

\subsection{Explicit Conversions}

\subsection{Implicit Conversions}
\label{sec:implicit-conversions}

\section{Workflows}
\label{sec:workflows}



