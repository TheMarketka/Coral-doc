%!TEX TS-program = xelatex
%!TEX encoding = UTF-8 Unicode

\chapter{The Coral Standard Library}

The Coral Standard Library (CSL\footnote{Not to be confused with CLS, which is this document.}) consists of the module \code{Lang~[coral]} with a number of classes and submodules. Some of these classes are described in the following sections. 

\section{Root Classes}
\label{sec:root-classes}

The root of the class hierarchy of Coral is formed by the class \code{Object}. On the other side, the bottom of the class hierarchy is formed by the class \code{Nothing}. The implicit contract here is that every class conforms to \code{Object}, and \code{Nothing} conforms to every class. 

All classes and types in Coral are accompanied by the \code{Any} type (\sref{sec:any-value-type}), for which every class conforms to it, pretty much like \code{Object}, but using it enables dynamic behavior, as this type is implicitly used for ``unknown'' types throughout Coral. 

User-defined Coral classes that do not explicitly inherit from any class inherit from \code{Object} implicitly. 

A part of the signatures of these root classes is described by the following definitions. 

\begin{lstlisting}
module Lang~[coral]

class Object 
    extends ()
  method equals (that: Any): Boolean := $\ldots$
  alias :'=' is :equals
  
  method to_string: String := $\ldots$
  method hash_code: Integer := $\ldots$
  
  method tap: self.type := { yield self; self }
end

sealed abstract class Nothing extends () {}
\end{lstlisting}






\section{Scalar Value Classes}
\label{sec:scalar-value-classes}

\subsection{Numeric Scalars}

Types \code{Number.Integer}, \code{Number.Real} and \code{Number.Complex} are class clusters for various underlying classes, representing various subrange classes, and each of them inherits from class \code{Number}. 

Partial order and ranking is defined on the numeric types as follows, using definitions of form $(T_1 \commadots T_n) \conforms_w T$, where $T_1 \commadots T_n$ weakly conforms to $T$. 
\begin{itemize}
\item \lstinline!(Integer_8, Integer_8_Unsigned) $\conforms_w$ Integer_16!
\item \lstinline!(Integer_8_Unsigned) $\conforms_w$ Integer_16_Unsigned!

\item \lstinline!(Integer_16, Integer_16_Unsigned) $\conforms_w$ Integer_32!
\item \lstinline!(Integer_16_Unsigned) $\conforms_w$ Integer_32_Unsigned!

\item \lstinline!(Integer_32, Integer_32_Unsigned) $\conforms_w$ Integer_64!
\item \lstinline!(Integer_32_Unsigned) $\conforms_w$ Integer_64_Unsigned!

\item \lstinline!(Integer_64, Integer_64_Unsigned) $\conforms_w$ Integer_128!
\item \lstinline!(Integer_64_Unsigned) $\conforms_w$ Integer_128_Unsigned!

\item \lstinline!(Integer_128, Integer_128_Unsigned) $\conforms_w$ Decimal!
\item \lstinline!(Integer_128_Unsigned) $\conforms_w$ Decimal_Unsigned!

\item \lstinline!(Decimal_Unsigned) $\conforms_w$ Decimal!
\item \lstinline!(Integer_128_Unsigned) $\conforms_w$ Decimal_Unsigned!

\item \lstinline!(Decimal, Decimal_Unsigned) $\conforms_w$ Integer!
\item \lstinline!(Decimal_Unsigned) $\conforms_w$ Integer_Unsigned!

\item \lstinline!(Float_32) $\conforms_w$ Float_64!
\item \lstinline!(Float_64) $\conforms_w$ Float_128!
\item \lstinline!(Float_128) $\conforms_w$ Decimal!

\item \lstinline!(Integer, Integer_Unsigned) $\conforms_w$ Real!
\item \lstinline!(Real) $\conforms_w$ Complex!
\end{itemize}

\code{Integer_8}, alias \code{Byte}, and \code{Integer_8_Unsigned}, alias \code{Byte_Unsigned}, are the lowest-ranked types in this order, whereas \code{Complex} is the highest-ranked. Ranking does not imply conformance (\sref{sec:conformance}), e.g. \code{Integer_32} is not a subtype of \code{Integer_64}. However, the object \code{Predef} (\sref{sec:predef}) defines views (\sref{sec:views}) from every numeric value to all higher-ranked numeric value types. 

The \code{Integer} type has also a shorter alias of \code{Int}. 





\subsection{The \code{Boolean} Class}

The \code{Boolean} class has only two member values, represented by the keywords \code{yes} and \code{no}. 





\subsection{The \code{Unit} Class}

The \code{Unit} class has only one value, represented by the construct \code{()}. 





\section{Standard Reference Classes}

This section presents some standard Coral reference classes, which are treated in a special way by the Coral compiler -- Coral provides a syntactic sugar for them. 




\subsection{The \code{String} Classes}

Coral's String has a literal defined for it. Strings in Coral are immutable, and a mutable version exists: \code{Mutable_String}. Both classes inherit from \code{String_Like}. 

\paragraph{Note}
Single characters in Coral are abstracted by strings of length exactly $1$. 

\subsection{The \code{Symbol} Class}


The Symbol class is intended to represent names throughout programs. As strings may or may not exist in multiple instances representing the exact same sequence of code points, there is always only up to one instance of a symbol that represents the exact same sequence of code points. Class names, method names and all other names are represented by symbol instances, created on demand if necessary. Symbols are also a good choice for dictionary keys. 





\subsection{The \code{Tuple} Classes}

Scala defines tuple classes ~\lstinline!Tuple_$n$!~ for $n = 2 \commadots +\infty$, using autoloading (\sref{sec:autoloading}). These are defined as follows:

\begin{lstlisting}
module Lang~[coral]
case class Tuple_$n$ [+T_1, $\ldots$, +T_$n$] (_1: T_1, $\ldots$ _$n$: T_$n$)
    extends Object
  def apply (1) := _1
  $\ldots$
  def apply ($n$) := _$n$
  
  def to_string := "(#{_1}, #{$\ldots$}, #{_$n$})"
end
\end{lstlisting}

The implicitly imported \code{Predef} object (\sref{sec:predef}) defines the names \code{Pair} as an alias of \code{Tuple_2} and \code{Triple} as an alias of \code{Tuple_3}. 






\subsection{The \code{Function} Traits}

Coral defines function traits ~\lstinline!Function_$n$!~ for $n = 2 \commadots +\infty$, using autoloading (\sref{sec:autoloading}). These are defined as follows: 

\begin{lstlisting}
module Lang~[coral]
trait Function_$n$ [-T_1, $\ldots$, -T_$n$, +R]
    extends Object
  message apply (x_1: T_1, $\ldots$, x_$n$: T_$n$): R
  def to_string := 'Function_$n$'
end
\end{lstlisting}

A subtype of \code{Function_1} represents partial functions, which are undefined on some points in their arguments domain. In addition to the \code{apply} method of functions, partial functions also have a \code{defined_at?} message, which tells whether the function is defined at the given argument:

\begin{lstlisting}
module Lang~[coral]
trait Partial_Function [-A, +B]
    extends Function_1[A, B]
  message defined_at? (x: A): Boolean
  def to_string := 'Partial_Function'
end
\end{lstlisting}





\section{The \code{Predef} Object}
\label{sec:predef}

The \code{Predef} object defines some usual standard functions and type aliases for Coral programs. It is always implicitly imported, so that all its defined members are available without qualification. The \code{Predef} object may not be implicitly imported in a scope where ~\lstinline!pragma no-predef! is applied. 

Some of its members are defined as follows:
\begin{lstlisting}
module Lang~[coral]
object Predef extends Object

  // identity functions
  def identity [A] (x: A): A := x
  def implicitly [T] (implicit e: T) := e

  // tuples
  type Pair [+A, +B] := Tuple_2[A, B]
  type Triple [+A, +B, +C] := Tuple_3[A, B, C]

  // implicit conversions
  implicit def int_8_to_int_16 (i: Integer_8): Integer_16 := $\ldots$
  $\ldots$

  // math aliases
  type Number := Math.Number
  type Integer := Number.Integer
  type Real := Number.Real
  type Complex := Number.Complex
  type Int := Integer
  $\ldots$
  
  // reading, printing
  def print (x: Any) := VM.Console.print(x)
  def print_line (x: Any) := VM.Console.print_line(x)
  def printf (text: String_Like, *xs: Any) := VM.Console.printf(text, *xs)
  
  def read_line: String := VM.Console.read_line
  def read_boolean := VM.Console.read_boolean
  def read_byte := VM.Console.read_byte
  $\ldots$
  
  // definitions
  implicit final class Arrow_Assoc [A] (private val myself: A)
      extends Object
    operator => [B] (that: B): Tuple_2[A, B] := Tuple_2(myself, that)
  end class
\end{lstlisting}\begin{lstlisting}
end
\end{lstlisting}






