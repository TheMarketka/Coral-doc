%!TEX TS-program = xelatex
%!TEX encoding = UTF-8 Unicode

\newcommand{\Unicode}[1]{\mbox{$\backslash$u{#1}}}
\newcommand{\UnicodeRange}[2]{\mbox{$\backslash$u{#1}-$\backslash$u{#2}}}

\chapter{Lexical Syntax}

Coral programs are written using the Unicode character set; Unicode supplementary characters are supported as well. Coral programs are preferably encoded with the UTF-8 character encoding. While every Unicode character is supported, usage of Unicode escapes is encouraged, since fonts that IDEs might use may not support the full Unicode character set.

Grammar of lexical tokens is given in the following sections. These tokens are then used as terminal symbols of the semantical grammar. 

To construct tokens, characters are distinguished according to the following classes (Unicode general category given in parentheses):
\begin{itemize}
\item 
Whitespace characters. \Unicode{0020} | \Unicode{0009} | \Unicode{000D} | \Unicode{000A} (space, tab character, carriage return, line feed)

\item
Letters, which include lower case letters (Ll), upper case letters (Lu), title-case letters (Lt), other letters (Lo), letter numerals (Nl), modifier letters (Lm) and the two characters \Unicode{0024} `\lstinline[mathescape=false]!$!' and \Unicode{005F} `\lstinline!_!', which both count as upper case letters. 

\item
Digits `\lstinline!0!' | \ldots \thinspace | `\lstinline!9!'. 

\item
Parentheses. `\lstinline!(!' | `\lstinline!)!' | `\lstinline![!' | `\lstinline!]!' | `\lstinline!{!' | `\lstinline!}!'

\item
Delimiter characters. `\lstinline!'!' | `\lstinline!"!' | `\lstinline!.!' | `\lstinline!;!' | '\lstinline!,!'

\item
Operator characters. These consist of all printable ASCII characters \UnicodeRange{0020}{007F} that are in none of the sets above, mathematical symbols (Sm) and other symbols (So). 
\end{itemize}

\newpage






\section{Identifiers}
\label{sec:identifiers}

\syntax\begin{lstlisting}
op_id    ::= opchar {opchar}
var_id   ::= lower id_rest
plain_id ::= upper id_rest
           | var_id
id       ::= ['@' | '@@'] (plain_id
             | '`' [(id_elem - '(') {id_elem}] '`')
id_rest  ::= {letter | digit | '+' | '-' | '/' | '_'}
id_elem  ::= id_rest | printable_char
\end{lstlisting}

There are more kinds of identifiers. An identifier can start with a letter, which can be followed by an arbitrary sequence of letters, digits, underscores and operator characters. The identifier may be prefixed with one or two {\em at} ``\code{@}'' signs, creating an instance variable or a class object variable identifier, respectively. These forms are called {\em plain identifiers}. An identifier may also start with an operator character, followed by arbitrary sequence of operator characters, forming operator identifiers, which can only be used in expressions that directly involve operators (\sref{sec:prefix-infix-ops}). 

An identifier may also be formed by an identifier between back-quotes (``\lstinline!$\,$`$\,$!''), to resolve possible name clashes with Coral keywords, and to allow to identify operators. Instance variable names and class instance variable names never clash with a keyword name, since these are distinguished by the preceding ~``\lstinline!@!''~ and ~``\lstinline!@@!''~ respectively. 

Coral programs are parsed greedily, so that a longest match rule applies. Letters from the syntax may be any Unicode letters, but English alphabet letters are recommended, along with English names.

The ``\lstinline[mathescape=false]!$!'' character is reserved. 






\section{Keywords}\label{sec:keywords}

A set of identifiers is reserved for language features instead of for user identifiers. However, unlike in most other languages, keywords are not being recognized inside of paths, except for a few specific cases.

The following names are the reserved words.

\begin{lstlisting}
abstract      advice        advice-execution            after
alias         annotation    arguments     around        as
aspect        assert        atomic        begin         before
bitfield      break         broken        case          catch
class         clone         cloned        constant      constraint
constructor   declare       def           delta         destructor
digits        do            else          elsif         end
ensure        enum          execution     exhausted     extends
final         for           for-some      function      get
goto          handler       if            implements    implicit
in            include       interface     invariant     invoke
is            label         lazy          let           loop
macro         match         memoize       message       method
module        native        next          nil           no
not           object        of            opaque        operator
optional      origin        otherwise     out           override
pointcut      pragma        prepend       private       property
protected     protocol      public        raise         raising
range         record        redo          refine        refinement
rescue        retry         return        returning     requires
reverse       sealed        self          set           skip
soft          step          super         switch        target
then          this          throw         throwing      trait
transparent   type          undefined     unless        unowned
until         union         unit-of-measure             use
val           var           yes           weak          when
while         with          yield
\end{lstlisting}

Not every reserved word is a keyword in every context, this behavior will be further explained by syntax definitions. For example, the bitfield reserved word is only recognized as a keyword inside an enumeration definition context, in a specific place. In case of ambiguous syntax views of a sequence of source tokens, the ones that include keywords are preferred.\footnote{This may happen with selection sequences that would include e.g. ~\lstinline!prefix.type!. To treat \code{type} as another selection, enclose it in backticks: ~\lstinline[deletekeywords={type}]!prefix.`type`!.} In any case, reserved words enclosed in backticks (``\lstinline!$\,$`$\,$!'') are not treated as keywords. 






\section{Newline Characters}
\label{sec:newlinecharacters}

\syntax\begin{lstlisting}
semi ::= nl {nl} | ';'
\end{lstlisting}

Coral is a line-oriented language, in which statements are expressions and may be terminated by newlines, as well as by semi-colon operator. A newline in a Coral source file is treated as the special separator token \lstinline@nl@ if the following criteria are satisfied:

\begin{enumerate}
\item The token immediately preceding the newline can terminate an expression.
\item The first non-whitespace token immediately following the newline can continue the expression on the previous line. 
\end{enumerate}

Since Coral may be interpreted in a REPL\footnote{Read-Eval-Print Loop} fashion, there are no other suitable criteria. Such a token that can terminate an expression is, for instance, not a binary operator or a message sending operator, which both require further tokens to create an expression. Keywords that expect any following tokens also can not terminate expressions. Coral interpreters and compilers do not look-ahead beyond newlines.

If the token immediately preceding the newline can not terminate an expression and is followed by more than one newline, Coral still sees that as only a one significant newline, to prevent any confusion.

Keywords that can terminate an expression are: \lstinline@break@, \lstinline@end@, \lstinline@opaque@, \lstinline@native@, \lstinline@next@, \lstinline@nil@, \lstinline@no@, \lstinline@redo@, \lstinline@retry@, \lstinline@return@, \lstinline@self@, \lstinline@skip@, \lstinline@super@, \lstinline@this@, \lstinline@transparent@, \lstinline@yes@, \lstinline@yield@.

Since REPL can not foresee what the following line will hold, the second criterion can not be checked earlier than when the following line is entered. However, if the following line continues the previous expression, it is treated as such. The only exception is when the previous expression is of an assignment kind, in that case, there is a simple solution that need to be applied: enclose the right-hand side of the assignment in parentheses. That way the right-hand side expression is not ended sooner than before the closing parenthesis. Too keep syntax rules consistent across both REPL and pre-compiled programs, these rules apply to both. 

\example Different treating of assignment expression in presence of newline tokens. 
\begin{lstlisting}
// the expression:
val a := b
    .c
// is treated as:
(val a := b).c

// but the expression:
val a := (b
    .c)
// is treated as:
val a := b.c 
\end{lstlisting}

\example Different treating of $n$-ary expression in presence of newline tokens.
\begin{lstlisting}
// the expression:
val a := b
    ? c
    : d
// is treated as:
(val a := b).`?`(c) // and continues with an error,
                    // because `:` is not a legal operator

// but the expression:
val a := (b
    ? c
    : d)
// is treated as:
val a := b.`?`(c, d) 
\end{lstlisting}






\section{Operators}\label{sec:operators}
A set of identifiers is reserved for language features, some of which may be overridden by user space implementations. Operators have language-defined precedence rules that are supposed to usually comply to user expectations (principle of least surprise), and another desired precedence may be obtained by putting expressions with operators inside of parenthesis pairs. 

Binary (infix) operators have to be separated by whitespace or parentheses on both sides, unary operators by whitespace on left side -- the right side is what they are bound to. 






\section{Literals}\label{sec:literals}

There are literals for numbers (including integer, floating point and complex), characters, booleans, strings, symbols, regular expressions and collections (including tuples, lists, dictionaries and bags). 

\syntax\begin{lstlisting}
Literal ::= integer_literal
	      | floating_point_literal
	      | complex_literal
	      | character_literal
	      | string_literal
	      | symbol_literal
	      | regular_expression_literal
	      | Collection_Literal
	      | 'nil'
	      | 'undefined'
\end{lstlisting}






\subsection{Integer Literals}\label{sec:integerliterals}

\syntax\begin{lstlisting}
integer_literal ::= sign (
                    decimal_numeral
                  | hex_numeral
                  | octal_numeral
                  | binary_numeral)
decimal_numeral ::= digit {['_'] digit}
hex_numeral     ::= hex_prefix | hex_digit {['_'] hex_digit}
digit           ::= '0' | $\cdots$ | '9'
hex_digit       ::= '0' | $\cdots$ | '9' | 'a' | $\cdots$ | 'f'
octal_numeral   ::= oct_prefix oct_digit {'_' oct_digit}
oct_digit       ::= '0' | $\cdots$ | '7'
binary_numeral  ::= bin_prefix bin_digit {['_'] bin_digit}
bin_digit       ::= '0' | '1'
hex_prefix      ::= '0x'
bin_prefix      ::= '0b'
oct_prefix      ::= '0o'
sign            ::= ['+' | '-'] | ()
\end{lstlisting}

Integers are usually of type \lstinline@Number@, which is a class cluster of all classes that can represent numbers. Unlike Java, Coral supports both signed and unsigned integers directly. Usually integer literals that are obviously unsigned integers are automatically represented internally by a class that stores the integer unsigned, like \lstinline@Integer_64_Unsigned@. Math operations on numbers are handled internally in such a way that the user does't need to worry about the actual types of the numbers — when an integer overflow would occur, the result is stored in a larger container type. 

Underscores ``\code{_}'' are allowed between digits for readability, but are otherwise ignored. Decimal integer literals can begin with leading zeros ``\code{0}'', but those zeros are likewise ignored. 

Integral members of the \lstinline@Number@ class cluster include the following container types. 

\begin{enumerate}

  \item \lstinline@Integer_8@ ($-2^{7}$ to $2^{7}-1$), alias \lstinline@Byte@

  \item \lstinline@Integer_8_Unsigned@ ($0$ to $2^{8}$), alias \lstinline@Byte_Unsigned@

  \item \lstinline@Integer_16@ ($-2^{15}$ to $2^{15}-1$), alias \lstinline@Short@

  \item \lstinline@Integer_16_Unsigned@ ($0$ to $2^{16}$), alias \lstinline@Short_Unsigned@

  \item \lstinline@Integer_32@ ($-2^{31}$ to $2^{31}-1$)

  \item \lstinline@Integer_32_Unsigned@ ($0$ to $2^{32}$)

  \item \lstinline@Integer_64@ ($-2^{63}$ to $2^{63}-1$), alias \lstinline@Long@

  \item \lstinline@Integer_64_Unsigned@ ($0$ to $2^{64}$), alias \lstinline@Long_Unsigned@

  \item \lstinline@Integer_128@ ($-2^{127}$ to $2^{127}-1$), alias \lstinline@Cent@

  \item \lstinline@Integer_128_Unsigned@ ($0$ to $2^{128}$), alias \lstinline@Cent_Unsigned@

  \item \lstinline@Decimal@ ($-\infty$ to $\infty$)

  \item \lstinline@Decimal_Unsigned@ ($0$ to $\infty$)

\end{enumerate}

The special \lstinline@Decimal@ \& \lstinline@Decimal_Unsigned@ container types are also for storing arbitrary precision floating point numbers. All the container types are constants defined in the \lstinline@Number@ class and can be imported into scope if needed. 

Moreover, a helper type \lstinline@Number.Unsigned@ exists, which can be used for type casting in cases where an originally signed number needs to be treated as unsigned. 

Weak conformance applies to the inner members of \lstinline@Number@ class. 

For use with range types, \lstinline@Number.Integer@ and \lstinline@Number.Integer_Unsigned@ exist, to allow constraining of the range types to integral numbers.






\subsection{Floating \& Fixed Point Literals}
\label{sec:floatliterals}
\label{sec:fixedpointliterals}

\syntax\begin{lstlisting}
float_literal   ::= sign decimal_numeral '.' decimal_numeral
                    [exponent_part_e] [float_type]
                  | sign decimal_numeral exponent_part_e [float_type]
                  | sign decimal_numeral float_type
                  | sign hex_prefix hex_numeral '.' hex_numeral 
                    [exponent_part_p [float_type] 
                     | hex_exp float_type]
                  | sign hex_prefix hex_numeral exponent_part_p 
                    [float_type]
                  | sign hex_prefix hex_numeral hex_exp float_type
                  | sign oct_prefix octal_numeral '.' octal_numeral 
                    [exponent_part_e] [float_type]
                  | sign oct_prefix octal_numeral exponent_part_e 
                    [float_type]
                  | sign oct_prefix octal_numeral float_type
                  | sign bin_prefix binary_numeral '.' binary_numeral 
                    [exponent_part_e] [float_type]
                  | sign bin_prefix binary_numeral exponent_part_e 
                    [float_type]
                  | sign bin_prefix binary_numeral float_type
exponent_part_e ::= int_exp sign decimal_numeral
exponent_part_p ::= hex_exp sign decimal_numeral
int_exp         ::= 'e'
hex_exp         ::= 'p'
float_type      ::= 'f' | 'd' | 'q' | 'df'
\end{lstlisting}

Floating point and fixed point literals are of type \lstinline@Number@ as well as integral literals, and have fewer container types. Compiler infers the precision automatically, unless the \lstinline@float_type@ part is present. Literals that have \lstinline@float_type@ of ``\code{df}'' are ({\em d}ecimal) {\em f}ixed point literals. Also, floating point literals that are impossible to represent in binary form accurately are implicitly inferred to be fixed point literals, unless specifically converted to a floating point type or using a \code{float_type} of either \code{f} or \code{d}. From \code{Number}'s user perspective, this is only an implementation detail. 

\begin{enumerate}

  \item \lstinline@Float_32@ (IEEE 754 32-bit precision), alias \lstinline@Float@. 

  \item \lstinline@Float_64@ (IEEE 754 64-bit precision), alias \lstinline@Double@.

  \item \lstinline@Float_128@ (IEEE 754 128-bit precision).

  \item \lstinline@Decimal@ ($-\infty$ to $\infty$).

  \item \lstinline@Decimal_Unsigned@ ($0$ to $\infty$).

\end{enumerate}

Letters in the exponent type, hexadecimal numbers and float type literals have to be lower-case in Coral sources, but functions that parse floating point numbers do support them being upper-case for compatibility. 





\subsection{Not A Number}
\label{sec:nan}

A member named \code{Not_a_Number} exists in the class cluster \code{Number}, with an alias of \code{NaN}, to denote results of operations on numbers that are not numbers, e.g., result of division by zero. There is no literal for this special value in Coral. 






\subsection{Imaginary Number Literals}
\label{sec:imaginaryliterals}

\syntax\begin{lstlisting}
imaginary_literal   ::= real_number_literal 'i'
complex_literal     ::= [real_number_literal ('+' | '-') ]
                        imaginary_literal
	                  | imaginary_literal ('+' | '-') 
	                    real_number_literal
real_number_literal ::= integer_literal | float_literal
\end{lstlisting}

Imaginary number literals are of type \code{Number}, and have a basically a single container type: \code{Number.Complex}. The syntax requirement here is whitespace around the ``\code{+}'' and ``\code{+}'' signs, separating the real part from the imaginary part. Newlines as whitespace have the same effect as defined for cases where the sign could be considered to be an operator (\sref{sec:newlinecharacters}).





\subsection{Rational Number Literals}
\label{sec:rationalliterals}

\syntax\begin{lstlisting}
rat_suffix_literal  ::= real_number_literal 'r'
rational_literal    ::= real_number_literal '/' 
                        rat_suffix_literal
number_literal      ::= real_number_literal
	                  | imaginary_literal
	                  | complex_literal
	                  | rational_literal
\end{lstlisting}

Rational number literals are of type \code{Number}, and have a container type \code{Number.Rational}, which has further methods of creating instances of itself. There are also methods in \code{Number}, such as \code{/-}, which results in a rational number. Rational numbers have sort of increased accuracy of operations, especially when component types are integral types. 

\example Some literal notations and methods that result in rational numbers. 
\begin{lstlisting}
val a := 1 / 3r
val b := 1 /- 3
val c := Rational(1, 3)
\end{lstlisting}





\subsection{Units of Measure}
\label{sec:unitsofmeasuresyntax}

Coral has an addition to number handling, called {\em units of measure} (\sref{sec:units-of-measure}). Number instances can be annotated with a unit of measure to ensure correctness of arithmetic operations. 

\syntax\begin{lstlisting}
annotated_number ::= number_literal '[<' uom_expr '>]'
uom_expr         ::= Unit_Conv {',' Unit_Conv}
\end{lstlisting}

\example Some number literals annotated with units of measure:
\begin{lstlisting}
var min_motorway_speed := 90 [<km/hour>]
// in the end, same as:
var distance := 90 [<km>]
var time := 1 [<hour>]
min_motorway_speed := distance /- time
\end{lstlisting}






\subsection{Character Literals}
\label{sec:characterliterals}

\syntax\begin{lstlisting}
printable_char    ::= ? all visible and printable UTF-8 characters ? 
character_literal ::= '%'' (printable_char | char_escape_seq) '''
\end{lstlisting}

Character literals are of type \code{Character}, and are internally similar to strings (\sref{sec:stringliterals}) of length of 1 character. 

Character escape sequences are used in strings as well, and are defined as follows: 

\syntax\begin{lstlisting}[language=]
char_escape_seq ::= 
      '\'' (* Literal single-quote: ' *)
    | '\"' (* Literal double-quote: " *)
    | '\?' (* Literal question mark: ? *)
    | '\\' (* Literal backslash: \ *)
    | '\#' (* Literal hash: # *)
    | '\{' (* Literal left curly brace: { *)
    | '\}' (* Literal right curly brace: } *)
    | '\0' (* Binary zero (NUL, U+0000) *)
    | '\a' (* BEL (alarm) character (U+0007) *)
    | '\b' (* Backspace (U+0008) *)
    | '\f' (* Form feed (FF, U+000C) *)
    | '\n' (* End-of-line (U+000A) *)
    | '\r' (* Carriage return (U+000D) *)
    | '\t' (* Horizontal tab (U+0007) *)
    | '\v' (* Vertical tab (U+000B) *)
    | '\x$nn$' (* Byte value in hexadecimal *) 
    | '\$n$' (* Byte value in octal *)
    | '\$nn$' (* Byte value in octal *)
    | '\$nnn$' (* Byte value in octal *)
    | '\u$nnnn$' (* Unicode character U+$nnnn$ *)
    | '\u$nnnnnnnn$' (* Unicode character U+$nnnnnnnn$ *)
    | '\' named_char ';' (* Named character entity *)
\end{lstlisting}

The named character reference (\code{named_char}) is defined by HTML5 (\url{http://www.w3.org/TR/html5/syntax.html#named-character-references});






\subsection{Boolean Literals}
\label{sec:booleanliterals}

\syntax\begin{lstlisting}
boolean_literal ::= 'yes' | 'no'
\end{lstlisting}

Both literals are members of type \lstinline@Boolean@. The \lstinline@no@ literal has also a special behavior when being compared to \lstinline@nil@: \lstinline@no@ equals to \lstinline@nil@, while not actually being \lstinline@nil@. Identity equality is indeed different, and \code{no} does not match in pattern matching (\sref{sec:pattern-matching}) as \code{nil} and vice versa. The implication is that both \lstinline@nil@ and \lstinline@no@ are false conditions in \lstinline@if@-expressions. 





\subsection{String Literals}
\label{sec:stringliterals}

\syntax\begin{lstlisting}
string_literal      ::= int_string_literal
                      | raw_string_literal
int_string_literal  ::= '"' {int_string_element} '"'
                      | '%'  [str_flags] '"' {int_string_element} '"'
raw_string_literal  ::= '%r' [str_flags] '"' {string_element} '"'
string_element      ::= printable_char | char_escape_seq
int_string_element  ::= string_element | interpolated_expr
interpolated_expr   ::= '#{' Expr '}'
str_flags           ::= 'm' | 'i'
\end{lstlisting}

String literals are members of the type ~\lstinline@String@. Double quotes in interpolable string literals have to be escaped (\lstinline@\"@).

String literals appear in multiple forms:
\begin{itemize}
  \item {\em Raw strings} are of the form ~\lstinline!%r$f$"$s$"!. There are no interpolated expressions. If a sequence of string elements appears to be an interpolated expression, it is not, and is instead treated as a part of the raw string as it is. 
  \item {\em Interpolable strings} are of the forms ~\lstinline!"$s$"!~ and ~\lstinline!%$f$"$s$"!. Interpolated expressions can appear there, if its introducing character is not escaped by an odd number of backslashes. Interpolated expressions are evaluated, converted to \code{String_Like} and their result inserted at each place in the string where they appear, safely. 
  \item {\em Immutable strings} are all strings that do not carry the flag \code{m} (\textbf{m}utable). The flag \code{i} (\textbf{i}mmutable) is redundant in this manner, and is only provided for completeness. 
  \item {\em Mutable strings} are only those strings that carry the flag \code{m}. Immutable strings may be converted to mutable strings and vice versa by use of appropriate methods. 
\end{itemize}

In each form, $s$ is the string content, and $f$ are string flags, as defined. 





\subsection{Symbol Literals}
\label{sec:symbolliterals}

\syntax\begin{lstlisting}
symbol_literal       ::= simple_symbol | quoted_symbol
simple_symbol        ::= ':' plain_id
quoted_symbol        ::= ':"' {int_string_element} '"'
\end{lstlisting}

Symbol literals are members of the type \lstinline@Symbol@. They differ from \nameref{sec:stringliterals} in the way runtime handles them: while there may be multiple instances of the same string, there is always up to one instance of the same symbol. Unlike in Ruby, they do get released from memory when no code references to them anymore, so their object id (sometimes) varies with time. Coral does not require their ids to be constant in time. 









\subsection{Regular Expression Literals}\label{sec:regexpliterals}

\syntax\begin{lstlisting}
regexp_literal     ::= '%/' regexp_content_int '/' [regexp_flags]
	                 | '%r/' regexp_content_int '/' [regexp_flags]
	                 | '%r#' regexp_content '#' [regexp_flags]
	                 | '%r~' regexp_content_int '~' [regexp_flags]
regexp_content_int ::= regexp_element_int {regexp_element_int}
regexp_element_int ::= string_element | int_string_element
regexp_content     ::= string_element {string_element}
regexp_flags       ::= printable_char {printable_char}
\end{lstlisting}

Regular expression literals are members of the type \lstinline@Regular_Expression@ with alias of \lstinline@Regexp@. 






\subsection{Collection Literals}\label{sec:collectionliterals}

Collection literals are paired syntax tokens and as such, they are a kind of parentheses in Coral sources. 

\syntax\begin{lstlisting}
Collection_Literal ::= Tuple_Literal
	                 | List_literal
	                 | Dictionary_Literal
	                 | Bag_Literal
Tuple_Literal      ::= '(' [Exprs] ')'
List_Literal       ::= '%' Collection_Flags '[' [Exprs] ']'
Dictionary_Literal ::= '%' Collection_Flags '{' [Dict_Exprs] '}'
Bag_Literal        ::= '%' Collection_Flags '(' [Exprs] ')'
Dict_Exprs         ::= Dict_Expr {',' Dict_Expr}
Dict_Expr          ::= Simple_Expr1 '=>' Expr
	                 | id ':' Expr
Collection_Flags   ::= printable_char {printable_char}
\end{lstlisting}

Tuple literals are members of the \lstinline@Tuple@ type family. List literals are members of the \lstinline@List@ type, usually \lstinline@Array_List@ with alias of \lstinline@Array@. Dictionary literals are members of the \lstinline@Dictionary@ type with alias of \lstinline@Map@, usually \lstinline@Hash_Dictionary@ with alias of \lstinline@Hash_Map@. Bag literals are members of the \lstinline@Bag@ type, usually \lstinline@Hash_Bag@ or \lstinline@Hash_Set@. Collection flags may change the actual class of the literal, along with some other properties, described in the following text. 

List literal collection flags: 

\begin{enumerate}
\item Flag \lstinline@i@ = \textbf{i}mmutable, makes the list frozen. 
\item Flag \lstinline@l@ = \textbf{l}inked, makes the list a member of \lstinline@Linked_List@. 
\item Flag \lstinline@w@ = \textbf{w}ords, the following expressions are treated as words, converted to strings for each word separated by whitespace. 
\end{enumerate}

Dictionary literals collection flags:

\begin{enumerate}
\item Flag \lstinline@i@ = \textbf{i}mmutable, makes the dictionary frozen. 
\item Flag \lstinline@l@ = \textbf{l}inked, makes the dictionary a member of \lstinline@Linked_Hash_Dictionary@ (also has alias \lstinline@Linked_Hash_Map@).
\item Flag \lstinline@m@ = \textbf{m}ulti-map, the dictionary items are then either the items themselves, if there is only one for a particular key, or a set of items, if there is more than one item for a particular key. The dictionary is then a member of \lstinline@Multi_Hash_Dictionary@ (alias \lstinline@Multi_Hash_Map@) or \lstinline@Linked_Multi_Hash_Dictionary@ (alias \lstinline@Linked_Multi_Hash_Map@). 
\end{enumerate}

Bag literal collection flags:

\begin{enumerate}
\item Flag \lstinline@i@ = \textbf{i}mmutable, makes the bag frozen. 
\item Flag \lstinline@s@ = \textbf{s}et, the collection is a set instead of a bag (a specific bag, such that for each item, its tally is always $0$ or $1$, thus each item is in the collection up to once). 
\item Flag \lstinline@l@ = \textbf{l}inked, makes the collection linked, so either a member of \lstinline@Linked_Hash_Bag@ in case of a regular bag, or \lstinline@Linked_Hash_Set@ in case of a set. 
\end{enumerate}

Linked collections have a predictable iteration order in case of bags and dictionaries, or are simply stored differently in case of lists. 

The type of elements of a collection is inferred from combination of the expected type of the whole literal, expected type of each element and type argument of the collection literal. 
\begin{itemize}
  \item If an expected type is given, its type arguments take precedence over types of elements. Elements may be subject to implicit conversions (\sref{sec:implicit-conversions}). This inference works with an implicit conversion from a builder type to the target collection type -- there must exists such a builder type that has a build method returning the type that is the expected type or a type that is convertible to the expected type.\footnote{If a custom type is used, it has to statically conform to one of the protocols that the particular collection requires, so that the matching type argument can be retrieved and used for the collection builder.} 
  \item If a type argument is present, it takes precedence over both types of elements and the expected type of the literal, which is then used to implicitly convert the whole parameterized collection type. 
  \item If the expected type is undefined and no type argument is given, then the elements type is inferred as the least upper bound of all expected types of each element. If no element is given, then \code{Any} is inferred.\footnote{\code{Any} is inferred so that the shortest type argument-less collection literal would be more useful than just allowing elements of type \code{Nothing}, for which just one instance exists.}
  \item If the collection literal is a dictionary, then the least upper bound is computed for key and value types independently.
\end{itemize}

\example The following lines show how type inference works with collection literals. 
\begin{lstlisting}
val a: Array[String] := %[]  // Array[String]
val b := %[][String]         // Array[String]
val c := %["hello", "world"] // Array[String]

val d: Map[String, String] := %{} // Map[String, String]
val e := %{}[String, String]      // Map[String, String]
val f := %{"hello" => "world"}    // Map[String, String]

val g: Bag[String] := %()    // Bag[String]
val h := %()[String]         // Bag[String]
val i := %("hello", "world") // Bag[String]
\end{lstlisting}






\section{Whitespace \& Comments}\label{sec:whitespacecomments}

Tokens may be separated by whitespace characters and/or comments. Comments come in two forms: 

A single-line comment is a sequence of characters that starts with \lstinline@//@ and extends to the end of the line. 

A multi-line comment is a sequence of characters between \lstinline@/* and */@. Multi-line comments may be nested. 

Documentation comments are multi-line comments that start with \lstinline@/*!@. 





