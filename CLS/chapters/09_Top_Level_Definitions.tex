%!TEX TS-program = xelatex
%!TEX encoding = UTF-8 Unicode

\chapter{Top-Level Definitions}

\section{Compilation Units}
\label{sec:compilation-units}

\subsection{Modules}
\label{sec:modules}

\syntax\begin{lstlisting}
Compilation_Unit ::= {'module' Module_Path semi [Top_Stat_Seq]} 
                     Top_Stat_Seq
Top_Stat_Seq     ::= Top_Stat {semi Top_Stat}
Top_Stat         ::= {Annotation} {Modifier} Tmpl_Def
                   | Use
                   | Packaging
                   | Module_Object
                   | Expr
                   | ()
\end{lstlisting}

Module definitions are objects that have one main purpose: to join related code and separate it from the outside. Coral's approach to modules solves these issues: 
\begin{itemize}
\item {\em Namespaces}. A class with a name $C$ may appear in a module $M$ or a module $N$, or any other module, and yet be a different object. Modules may be nested.
\item {\em Vendor packages}. Even modules of the same name may co-exists, provided that they have a different vendor, which is just an identifier that looks like a reverse domain name (similar to Java or Scala packages). 
\end{itemize}

A compilation unit (a single source file) consists of a sequence of packagings, import clauses, and class and object definitions, which may be preceded (and should be preceded) by a module clause. 

A compilation unit 
\begin{lstlisting}
module $p_1$
$\ldots$
module $p_n$
$\stats$
\end{lstlisting}
starting with one or more module clauses is equivalent to a compilation unit consisting of the packaging
\begin{lstlisting}
module $p_1$
  $\ldots$
  module $p_n$
    $\stats$
  end module
end module
\end{lstlisting}

Implicitly imported into every compilation unit are, in that order: 
\begin{enumerate}
\item the module ~\lstinline!Root~Lang~[com.coral-lang]! 
\item the object ~\lstinline!Root~Lang~[com.coral-lang].Predef!
\end{enumerate} 
Members of a later import in that order hide members of an earlier import. 

% TBD: notice that the implicit import may be overridden by a pragma at the beginning of a compilation unit

The implicitly added code looks like the following code listing, with all its implications:\footnote{The \code{Root} is actually redundant, as explained in (\sref{sec:module-refs}).}
\begin{lstlisting}
use Root~Lang~[com.coral-lang].{_}
use Root~Lang~[com.coral-lang].Predef.{_}
\end{lstlisting}





\subsection{Packagings}

\syntax\begin{lstlisting}
Packaging  ::= 'module' Module_Path (Packaging1 | Packaging2)
Packaging1 ::= semi Top_Stat_Seq 'end' ['module']
Packaging2 ::= '{' Top_Stat_Seq '}'
\end{lstlisting}

A module is a special object which defines a set of member classes, objects and another modules. Like open templates (\sref{sec:open-templates}), modules are introduced by multiple definitions across multiple source files.  

A packaging ~\lstinline!module $p$ { $\stats$ }!~ or ~\lstinline!module $p$~[$v$] $\stats$ end!~ injects all definitions in \stats as members into the module whose qualified name is $p$. Members of a module are called {\em top-level} definitions. If a definition in \stats is labeled \code{private}, it is visible only for other members in the same module. 

Inside the packaging, all members of package $p$ are visible under their simple names. This rule extends to members of the enclosing modules of $p$ that are of the same {\em vendor}. However, every other module needs to either import the members with a use clause (\sref{sec:use-clauses}), or refer to it via its fully qualified name. 

The special \code{Root} ``module'' can only be specified as the first element of each packaging name. 

\example Given the packagings
\begin{lstlisting}
module A~[org.net] {
  module B~[org.net] {
    $\ldots$
  }
  module B~[org.net.prj] {
    $\ldots$
  }
}
module C~[org.net] {}
module D~[org.net.prj] {}
\end{lstlisting}
all members of the module ~\lstinline!B~[org.net]!~ are visible under their simple names to the modules ~\lstinline!B~[org.net]!~ and ~\lstinline!A~[org.net]!, but not the others: module \code{C} has the same vendor, but is located outside of the packaging of module ~\lstinline!B~[org.net]!, and module \code{D} is completely out of the packaging game.\footnote{The packaging game is too strong for the module \code{D}.} All members of the module ~\lstinline!B~[org.net.prj]!~ are visible under their simple names to the module ~\lstinline!B~[org.net.prj]!, but not the other modules. Since the module ~\lstinline!B~[org.net.prj]!~ is nested in the module ~\lstinline!A~[org.net]!, all of its members are not visible to members of a potential module ~\lstinline!A~[org.net.prj]!~, since it is not nested in it. 

The fully qualified names of these modules are as follows: 
\begin{itemize}
\item \lstinline!A~[org.net]!
\item \lstinline!A~[org.net].B~[org.net]!, same as ~\lstinline!A~[org.net].B! (\sref{sec:module-refs})
\item \lstinline!A~[org.net].B~[org.net.prj]!
\item \lstinline!C~[org.net]!
\item \lstinline!D~[org.net.prj]!
\end{itemize}

Notice how these fully qualified names do not use the ~\lstinline!Root~! path prefix, explained in (\sref{sec:module-refs}).

Selections ~\lstinline!$p$.$m$!~ from $p$ as well as imports from $p$ work as for objects. Moreover, unlike in Scala, modules may be used as values, instances of \code{Module} class, which shares some behavior with \code{Class} class. It is illegal to have a module with the same fully qualified name (minus the vendor parts) as a class or a trait. 

Top-level definitions outside a packaging are assumed to be injected into the \code{Object} class directly, and therefore visible to each other without qualification. However, as \code{Object} is actually a simple name for the fully qualified name ~\lstinline!Root~Lang~[com.coral-lang].Object!, no member is ever defined outside of packaging -- it may only seem to be so: the type of \code{self} pseudo-variable in ``global'' context (outside of any packagings) is ~\lstinline!Root~Lang~[com.coral-lang].Object!---a special instance of \code{Object} dedicated to handling ``global'' space---unless the source file is loaded in context of another instance, used with DSLs. 






\subsection{Module Object}

\syntax\begin{lstlisting}
Module_Object ::= 'module' 'object' Trait_Template
\end{lstlisting}

A module object ~\lstinline!module object $p$ extends $t$! can specify some properties of the module object, add new traits to it, and adds the members of the template $t$ to the module object $p$. There can be only one module object per module, but the module object definition is still an open template (\sref{sec:open-templates}). The module object has to have the leading template defined in a file named ~\lstinline!$p$.coral!~ in the module's root directory, otherwise the module object has an implied empty template. 

The module object should not define a member with the same name as one of the top-level objects or classes defined in module $p$. If there is a name conflict, it is an error. 

The module object has also a special role in defining entry points of the module. An entry point is a method of the module object that can be invoked from the outside, to actually run the module as a program (\sref{sec:programs}). 






\subsection{Module References}
\label{sec:module-refs}

\syntax\begin{lstlisting}
Path ::= Module_Path
\end{lstlisting}





\section{Programs}
\label{sec:programs}

A {\em program} is a module that has 1 or more entry points. An entry point is a method of the module object that can be invoked from the outside, to actually run the module as a program. To mark a method as an explicit entry point, use the \code{entry} pseudo-modifier\footnote{Actually, it is implemented as a method of the class \code{Module}, therefore not a keyword, but IDEs may opt-in to highlight it as such, due to its importance.} before the method definition or declaration. An implicit entry point is a method with a name \code{main} and a method type ~\lstinline!(Sequence[String]) $\mapsto$ Unit!. A module does not need to have any entry points at all -- that renders it a ``library-only'' module. 

\example The following example will create a hello world program by defining a module entry point in module \code{Test}. 

\syntax\begin{lstlisting}[morekeywords={entry}]
module Test ~[com.example]
module object {
  entry def main (args: Sequence[String]) := {
    Console.print_line "Hello world!"
  }
}
\end{lstlisting}

This program can be started by the command
\begin{lstlisting}
% coral Test
\end{lstlisting}








