%!TEX TS-program = xelatex
%!TEX encoding = UTF-8 Unicode

\chapter{Implicit Parameters \& Views}
\label{sec:implicit-params-views}

\section{The Implicit Modifier}

\syntax\begin{lstlisting}
Local_Modifier ::= 'implicit'
Param_Clauses  ::= {Param_Clause} '(' 'implicit' Params ')'
\end{lstlisting}

Template members and parameters labeled with \code{implicit} modifier can be passed to implicit parameters (\sref{sec:implicit-parameters}) and can be used as implicit conversions called views (\sref{sec:views}). 

\example The following code defined an abstract class of monoids and two concrete implementations, \code{String_Monoid} and \code{Int_Monoid}. The two implementations are marked implicit and will be used throughout the following discussions. 
\begin{lstlisting}
abstract class Monoid [A] extends Semi_Group [A] {
  def unit: A end
  def add (x: A, y: A): A end
}
object Monoids {
  implicit object String_Monoid extends Monoid[String] {
    def unit: String := ""
    def add (x: String, y: String): String := x + y
  }
  implicit object Int_Monoid extends Monoid[Integer] {
    def unit: Integer := 0
    def add (x: Integer, y: Integer): Integer := x + y
  }
}
\end{lstlisting}






\section{Implicit Parameters}
\label{sec:implicit-parameters}

An implicit parameter list ~\lstinline!(implicit $p_1 \commadots p_n$)!~ of a method marks the parameters $p_1 \commadots p_n$ as implicit. A method or constructor can have at most one implicit parameter list, and it must be the last parameter list given. 

A method with implicit parameters can be applied to arguments just like normal method. In this case the \code{implicit} label has no effect. However, if such a method misses arguments for its implicit parameters (determined by a missing consecutive function application -- \sref{sec:partial-applications}), such arguments will be automatically provided, if possible. 






\section{Views}
\label{sec:views}

\section{View Bounds}

\syntax\begin{lstlisting}
Type_Param ::= (id | '_') [Type_Param_Clause]
               ['>:' Type] ['<:' Type]
               {'<%' Type} {':' Type}
             | '<' (id | '_') '>' ['<:' id]
\end{lstlisting}


