%!TEX TS-program = xelatex
%!TEX encoding = UTF-8 Unicode

\documentclass[11pt,a4paper]{report}

\usepackage{../tex/coralstyle}

\begin{document}

\title{The Coral Language Specification}
\author{Markéta Nikola Lisová}
\maketitle

\begin{abstract}

Coral is a Ruby-like programming language which enhances advanced object- oriented programming with elements of functional programming. Every value is an object, in this sense it is a pure object-oriented language. Object blueprints are described by classes. Classes can be composed in multiple ways – classic inheritance, mixin composition, union and compound types.

Coral is also a functional language in the sense that every function is also an object. Therefore, function definitions can be nested and higher-order functions are supported out-of-the-box. Coral also has a limited support for pattern matching, which can emulate the algebraic types used in other functional languages.

Coral has been developed from 2012 in a home environment out of pure enthusiasm for programming and out of a desire for a truly versatile language. This document is a work in progress and will stay that way forever. It acts as a reference for the language definition and some core library classes.
\end{abstract}

\end{document}
