%!TEX TS-program = xelatex
%!TEX encoding = UTF-8 Unicode

\chapter[Introduction to The Amlantis System Language]{Introduction to \\The Amlantis System Language}

\minitoc

\newpage

% TBD: add a proper introduction

\section{About The Amlantis System Language}

\AmlSystem is a language that targets a more low-level runtime: the System Runtime. It has syntax equivalent to that of \Aml. 

\AmlSystem is different from \Aml in their libraries, along with minor language differences, which are described in this part. 

The main purpose of \AmlSystem is to be a language in which a dynamic runtime and compiler for \Aml can be implemented, which would make \Aml almost self-hosted. This specification does not require implementation of \AmlSystem to be self-hosted in any way, and the same applies to \Aml.

The System Runtime is nothing but a fancy name for whichever operating system is hosting the implementation of \AmlSystem. It can talk directly to the OS, it can indeed make use of \code{libc}, \code{libc++}, \code{libstdc++}, or any other library that would make the talking easier. 





\section{Stock Implementations}

The ``stock'' implementation of \AmlSystem is a self-hosted one, using bootstrapping.

\begin{itemize}
  \item \code{Stage 0}~ \AmlSystem compiler is written in a different language\footnote{At the time of writing this, Rust 1.6 was chosen.}, and its sole purpose is to be able to compile a subset of \AmlSystem features -- a particular subset needed to compile a \code{stage 1} compiler. 
  \item \code{Stage 1}~ \AmlSystem compiler is written in a subset of the \AmlSystem language, and written so that it can compile every feature of \AmlSystem that \code{stage 2} compiler. 
  \item \code{Stage 2}~ \AmlSystem compiler is a self-hosted compiler, written in \AmlSystem without limitations (\code{stage 1} compiler must compile every feature that it uses), and can compile every \AmlSystem program, including its full library and indeed, itself. Which is what makes it self-hosted. 
  \item \code{Stage 3}~ \AmlSystem compiler is practically the same compiler as \code{stage 2}, and is just compiled by itself, no extra code is required for this compiler, it's just a name for the output of \code{stage 2} applied to its own source. 
\end{itemize}

The specification of \AmlSystem only requires a \code{stage 3}~-like compiler, not necessarily self-hosted. The existence of the previously mentioned stages is not mandatory. 

The ``stock'' implementation of \Aml and \AmlSE is an almost self-hosted one, using basically the same language (\AmlSystem), but targeting a different runtime, which it implements.\footnote{Originally, C++14 and Rust were chosen as the most likely candidates, but both failed, thus leading to creation of \AmlSystem.}






