%!TEX TS-program = xelatex
%!TEX encoding = UTF-8 Unicode

\chapter*{A Brief Introduction to the Amlantis System}
\label{sec:brief-intro}

Amlantis System is a collection of specifications of programming and data definition languages and their related tools, runtimes and libraries. 

Those specifications do not always require particular implementations, but attempt to avoid undefined behaviours as much as possible. 

Amlantis also provides an open source implementation of those specifications, but it is indeed possible for other people to write their own implementations or forks\footnote{Forks and pull requests are preferred!} of this default implementation, probably focusing on optimizing other aspects of the system, maybe exploring new options of future development to be pull-requested into the default implementation. 





\section*{A Few Notes on the Name}

Amlantis' name has quite some history. The project started being named {\em Coral}, but that collided with another language of a similar name, \href{https://en.wikipedia.org/wiki/Coral_66}{{\em CORAL 66}}. Then it got renamed to {\em Gear}, but that again collided with another language of the same name, which seemed inactive at the time, \href{https://github.com/zippers/gear}{{\em zippers/gear}}. Than an idea was born and Aml was named {\em Amlantis}, which is whatever you want it to be. It could be a misspelling of {\em Atlantis}\footnote{Intentionally -- because otherwise, it would be named Atlantis, but there is already a city of that name.}, it could be an acronym like {\em A ML Language}, or maybe even something like {\em A ML Language And Neat Technology Improvement System}, or maybe {\em Caml} without the {\em C}. For the meaning of the cryptic {\em ML} part, search for the {\em Standard ML} or {\em OCaml}. 





\newpage

\section*{The Amlantis Languages}
\label{sec:amlantis-languages}

There is a couple of languages that come with a fully-featured Amlantis' implementation:

\begin{itemize}
  \item Virtual Machine Runtime languages:
    \begin{itemize}
      \item Amlantis Core (or \AmlCore, or even simpler \Aml\footnote{Whenever this document mentions \Aml alone, suffix-less, it mentions this core language, \AmlCore.}), the core programming language of the system. It is the most versatile language in this list, also allowing programming of DSLs etc. (\sref{part:language})
      \item Amlantis SE (or \AmlSE), also a functional-style focused language, but more {\em homoiconic}, using {\em Symbolic Expressions} and defining itself in its own data-structures (which are shared with the other \code{Aml} languages). (\sref{part:language-se})
    \end{itemize}
  \item System Runtime languages:
    \begin{itemize}
      \item Amlantis System (or \AmlSystem), based on \mbox{\lstinline!Aml/Core!}. (\sref{part:language-system})
    \end{itemize}
\end{itemize}

Languages are interoperable within the same runtime, and have interoperability tools across runtimes. This is due to different ABIs of each runtime. 







