%!TEX TS-program = xelatex
%!TEX encoding = UTF-8 Unicode

\chapter{Syntax}
\label{ch:aml-base-syntax}

The syntax of \AmlBase is based on Lisp-1, the so-called ``S-Expressions'' (shorter for ``symbolic expressions''\footnote{Which we will not abbreviate further as ``sexprs''.}) and blatantly borrowing from languages such as Scheme, Racket or Clojure, while maintaining similarities with the \Aml and \AmlCore languages. 




\section[The Aml/Base.Lang.Reader]{The \lstinline!Aml/Base.Lang.Reader!}
\label{sec:aml-base-lang-reader}




\subsection{Delimiters and Dispatch}
\label{subsec:aml-base-lang-reader-delimiters-dispatch}

Along with whitespace (as defined by Unicode ``White\_Space'' property), the following characters are delimiters:

\begin{lstlisting}
  ( ) [ ] { } " ' ` ~ ;
\end{lstlisting}

The \lstinline@#@ character has got a special meaning, determined by the following character or characters; see below for details. 

After skipping whitespace, the \AmlBase reader dispatches based on the next character or characters in the source port this way:

\begin{table}
\begin{tabular}{ r l }
  \lstinline!(! & starts a pair or list; see \nameref{subsec:aml-base-lang-reader-lists}. \\
  \lstinline![! & starts a pair, list, or array; see \nameref{subsec:aml-base-lang-reader-lists}. \\
  \lstinline!{! & starts a structure; see \nameref{subsec:aml-base-lang-reader-structures}. \\
  \lstinline!)! & matches \lstinline!(! or raises error. \\
  \lstinline!]! & matches \lstinline![! or raises error. \\
  \lstinline!}! & matches \lstinline!{! or raises error. \\
  
  \lstinline!"! & starts a string; see \nameref{subsec:aml-base-lang-reader-strings}. \\
  \lstinline!'! & starts a quote or a type variable; see \nameref{subsec:aml-base-lang-reader-quotes}. \\
  \lstinline!`! & starts a tag or symbol; see \nameref{subsec:aml-base-lang-reader-polytags} and \nameref{subsec:aml-base-lang-reader-symbols}. \\
  \lstinline!~! & starts a label; see \nameref{subsec:aml-base-lang-reader-labels}. \\
  \lstinline!;! & starts a line comment; see \nameref{subsec:aml-base-lang-reader-comments}. \\
  
  \lstinline!#y! & true; see \nameref{subsec:aml-base-lang-reader-booleans}. \\
  \lstinline!#n! & false; see \nameref{subsec:aml-base-lang-reader-booleans}. \\
  \lstinline!#t! or \lstinline!#T! & true; see \nameref{subsec:aml-base-lang-reader-booleans}. \\
  \lstinline!#f! or \lstinline!#F! & false; see \nameref{subsec:aml-base-lang-reader-booleans}. \\  
  
  \lstinline!#(! & starts a set; see \nameref{subsec:aml-base-lang-reader-sets}. \\
  \lstinline!#[! & starts a vector; see \nameref{subsec:aml-base-lang-reader-vectors}. \\
  \lstinline!#{! & starts a dictionary; see \nameref{subsec:aml-base-lang-reader-dictionaries}. \\
  
  \lstinline!#\! & starts a character; see \nameref{subsec:aml-base-lang-reader-characters}. \\
  \lstinline!#"! & starts a byte string; see \nameref{subsec:aml-base-lang-reader-strings}. \\
  \lstinline!#~! & starts a symbol; see \nameref{subsec:aml-base-lang-reader-symbols}. \\
  \lstinline!#:! & starts a keyword; see \nameref{subsec:aml-base-lang-reader-keywords}. \\
  \lstinline!#'! & starts a quote; see \nameref{subsec:aml-base-lang-reader-quotes}. \\
  \lstinline!#`! & starts a quasi-quote; see \nameref{subsec:aml-base-lang-reader-quotes}. \\
  \lstinline!#,! & starts a (possibly splicing) unquote; see \nameref{subsec:aml-base-lang-reader-quotes}. \\
  \lstinline!#s'! & starts a syntax quote; see \nameref{subsec:aml-base-lang-reader-quotes}. \\
  \lstinline!#s`! & starts a syntax quasi-quote; see \nameref{subsec:aml-base-lang-reader-quotes}. \\
  \lstinline!#s,! & starts a syntax (possibly splicing) unquote; see \nameref{subsec:aml-base-lang-reader-quotes}. \\
  \lstinline!#^! & starts a metadata; see \nameref{subsec:aml-base-lang-reader-metadata}. \\
  \lstinline!##! & starts a parameterized read; see \nameref{subsec:aml-base-lang-reader-parameterized-reads}. \\
  
  \lstinline!#<<! & starts a string; see \nameref{subsec:aml-base-lang-reader-strings}. \\
  
  \lstinline!#|! & starts a block comment; see \nameref{subsec:aml-base-lang-reader-comments}. \\
  \lstinline!#;! or \lstinline!#_! & starts an S-Expression comment; see \nameref{subsec:aml-base-lang-reader-comments}. \\
  ``\lstinline@#! @'' & starts a line comment; note that the space is necessary; see \nameref{subsec:aml-base-lang-reader-comments}. \\
  \lstinline@#!/@ & starts a line comment; see \nameref{subsec:aml-base-lang-reader-comments}. \\
  
  \lstinline@#!@ & may start a reader extension use; see \nameref{subsec:aml-base-lang-reader-extension}. \\
  \lstinline!#reader! & starts a reader extension use; see \nameref{subsec:aml-base-lang-reader-extension}. \\
  \lstinline!#lang! & starts a reader extension use; see \nameref{subsec:aml-base-lang-reader-extension}. \\
  
  \lstinline!#$other$! & starts a handler defined in current readtable or raises error. \\
  
  {\em otherwise} & starts a symbol; see \nameref{subsec:aml-base-lang-reader-symbols}. \\
\end{tabular}
\caption{\label{tab:aml-base-lang-reader-dispatch}Table of Dispatches recognized by \lstinline!Aml/Base.Lang.Reader!.}
\end{table}





\newpage

\subsection{Reading Symbols}
\label{subsec:aml-base-lang-reader-symbols}

A sequence of source characters that does not start with any of the sequences defined in \nameref{subsec:aml-base-lang-reader-delimiters-dispatch} is parsed as either a symbol or a number. The sequence ``\lstinline!#~!'' also starts a symbol. A successful number parse takes precedence over symbol parse, see \nameref{subsec:aml-base-lang-reader-numbers}. 

Symbols that consist entirely of ``operator'' characters (Unicode character classes Sm, So, except for \lstinline!0x0060!) are treated specially, TBD\footnote{This is yet to be properly defined, but generally a property will be added to the form marking it ``operator''.}.





\subsection{Reading Numbers}
\label{subsec:aml-base-lang-reader-numbers}





\subsection{Reading Booleans}
\label{subsec:aml-base-lang-reader-booleans}





\subsection{Reading Pairs, Lists \& Arrays}
\label{subsec:aml-base-lang-reader-lists}





\subsection{Reading Structures}
\label{subsec:aml-base-lang-reader-structures}





\subsection{Reading Strings}
\label{subsec:aml-base-lang-reader-strings}





\subsection{Reading Characters}
\label{subsec:aml-base-lang-reader-characters}





\subsection{Reading Quotes}
\label{subsec:aml-base-lang-reader-quotes}





\subsection{Reading Comments}
\label{subsec:aml-base-lang-reader-comments}





\subsection{Reading Vectors}
\label{subsec:aml-base-lang-reader-vectors}





\subsection{Reading Sets}
\label{subsec:aml-base-lang-reader-sets}





\subsection{Reading Dictionaries}
\label{subsec:aml-base-lang-reader-dictionaries}





\subsection{Reading Keywords}
\label{subsec:aml-base-lang-reader-keywords}





\subsection{Reading Polymorphic Tags}
\label{subsec:aml-base-lang-reader-polytags}





\subsection{Reading Labels}
\label{subsec:aml-base-lang-reader-labels}





\subsection{Reading Metadata}
\label{subsec:aml-base-lang-reader-metadata}





\subsection{Reading Parameterized Reads}
\label{subsec:aml-base-lang-reader-parameterized-reads}





\subsection{Reading via an Extension}
\label{subsec:aml-base-lang-reader-extension}

When the reader dispatches on the \lstinline!#reader! form, it recursively applies another reader to the current source port. 

% TODO: there is more supposed to be happening before a switch to another reader is made, like, consulting the current reader guard

First, the reader recursively reads the next datum after \lstinline!#reader!, and uses it as path to the another reader. Such reader is then loaded, and \lstinline!read! is used when this reader is in \lstinline!read! mode, or else, \lstinline!read-syntax! is used when this reader is in \lstinline!read-syntax! mode.

The \lstinline!#lang! reader form is similar. It must be followed by a single whitespace character (preferably a single space, ASCII 32), and then followed by an identifier form. The complete form is then terminated by a new line, or end-of-file. A sequence \lstinline!#lang $name$! is equivalent to \lstinline!#reader $name$.Lang.Reader!. 

% TODO: allow arguments for the readers? 

For compatibility with e.g. R\textsuperscript{6}RS, \lstinline@#!@ is an alias for \lstinline!#lang! followed by a space when it is followed by alphanumeric ASCII, \lstinline!+!, \lstinline!-! or \lstinline!_!. 





\subsubsection{S-Expression Reader Language}

\begin{lstlisting}
#lang s-exp $path$
\end{lstlisting}





\subsubsection{Chaining Reader Language}

\begin{lstlisting}
#lang reader $path$
\end{lstlisting}






