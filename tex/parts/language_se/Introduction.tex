%!TEX TS-program = xelatex
%!TEX encoding = UTF-8 Unicode

\chapter{Introduction to Amlantis SE}

\minitoc

\newpage

% TBD: add a proper introduction

\section{About The Amlantis SE Language}

\AmlSE is a language that targets the same runtime as \Aml. It has different syntax though, more close the Lisp language family. Technically, \AmlSE is an extension for the other \Aml languages, as it serves primarily as a tool for data definitions, which happen to be also executable. 

Like Lisps, \AmlSE uses \href{https://en.wikipedia.org/wiki/S-expression}{{\em Symbolic Expressions}}\footnote{This is also where its name comes from, for now.} for the data definitions, which is also similar to the serialization format of \href{https://github.com/janestreet/sexplib}{OCaml's \code{Sexp} extension}. The syntax is based on Scheme, but uses elements of \Aml's syntax -- including identifier syntax, which needs to be compatible with \Aml. 






