%!TEX TS-program = xelatex
%!TEX encoding = UTF-8 Unicode

\chapter{Top-Level Definitions}

\minitoc

\newpage

\section{Compilation Units}
\label{sec:compilation-units}

\subsection{Modules}
\label{sec:modules}

\grammar\begin{lstlisting}
Implementation_File
    ::= Implementation_Packaging
Signature_File
    ::= Signature_Packaging
Script_File
    ::= Implementation_File
Script_Fragments
    ::= Script_Fragment {';;' Script_Fragment}
Script_Fragment
    ::= Script_Stat
Implementation_Packaging 
    ::= {Pragma} 'module' Struct_Path Impl_Packaging
      | Anonymous_Implementation_Packaging
Signature_Packaging
    ::= {Pragma} 'module' 'signature' Struct_Path Sig_Packaging
Impl_Packaging
    ::= semi [Impl_Pkg_Seq] ['end']  
        (* 'end' is only optional if the module is alone in the file *)
Impl_Pkg_Seq
    ::= Impl_Pkg_Stat {semi Impl_Pkg_Stat}
Impl_Pkg_Stat 
    ::= Use_Clause
      | Expr
      | Struct_Def
Sig_Packaging
    ::= semi [Sig_Pkg_Seq] 'end'
Sig_Pkg_Seq 
    ::= Sig_Pkg_Stat {semi Sig_Pkg_Stat}
Sig_Pkg_Stat
    ::= Struct_Spec
      | Use_Clause
Anonymous_Implementation_Packaging
    ::= Impl_Pkg_Seq
        (* the name of the module is derived from the file's name *)
Script_Stat
    ::= Impl_Pkg_Stat
      | Sig_Pkg_Stat
\end{lstlisting}

Module definitions are objects that have one main purpose: to join related code and separate it from the outside. Aml's approach to modules solves these issues: 
\begin{itemize}
  \item {\em Namespaces}. A class with a name $C$ may appear in a module $M$ or a module $N$, or any other module, and yet be a different object. Modules may be nested.
  \item {\em Vendor packages}. Even modules of the same name may co-exists, provided that they have a different vendor, which is just another identifier. 
\end{itemize}

An {\em implementation file} consists of a sequence of packagings, import clauses, and class and object definitions, which are preceded by a module clause. A {\em signature file} consists of a sequence of packagings, import clauses, and structure elements specifications, including interfaces and protocols. 

Implicitly imported into every compilation unit are, in that order: 
\begin{enumerate}
\item the module ~\lstinline!Aml/Language! 
\item the submodule ~\lstinline!Aml/Language.Predefined!
\end{enumerate} 
Members of a later import in that order hide members of an earlier import. 

The implicit import of these modules can be disabled in each scope by a pragma. % TBD: when the exact name and location of the pragma is defined, fill it here

% TBD: notice that the implicit import may be overridden by a pragma at the beginning of a compilation unit

The implicitly added code looks like the following code listing, with all its implications:
\begin{lstlisting}
use global.Aml/Language.{_}
use global.Aml/Language.Predefined.{_}
\end{lstlisting}

The vendor specified in module's \code{src/module.aml} is inherited by every source file in the same module and it's submodules (unless explicitly overwritten). Modules imported from within the module object defined in \code{src/module.aml} also make those modules available without fully qualifying their name in other source files of the same module. 





\subsection{Packagings}

A module is a special object which defines a set of member classes, objects and another modules. Like open templates (\sref{sec:open-templates}), modules are introduced by multiple definitions across multiple source files.  

A packaging ~\lstinline!module $p$ { $\stats$ }!~ or ~\lstinline!module $v$/$p$ $\stats$ end!~ injects all definitions in $\stats$ as members into the module whose qualified name is $p$. Members of a module are called {\em top-level} definitions. If a definition in $\stats$ is labelled \code{private}, it is visible only for other members in the same module. 

Inside the packaging, all members of package $p$ are visible under their simple names. This rule extends to members of the enclosing modules of $p$ (that are of the same {\em vendor}). However, every other module needs to either import the members with a use clause (\sref{sec:use-clauses}), or refer to it via its fully qualified name. 

The special {\em vendor-less} \code{global} ``module'' (defined as a part of \code{Long_Id} in \sref{sec:type-paths}) can only be specified as the first element of each packaging name, and can be used to refer to the ``global'' scope of names, from within any nested scope.

\example Given the packagings
\begin{lstlisting}
module An_Org/A
  module B
    $\ldots$
  end
end
module An_Org/C end
module Another_Org/D end
\end{lstlisting}
all members of the module ~\lstinline!An_Org/A.B!~ are visible under their simple names to the modules ~\lstinline!An_Org/A.B!~ and ~\lstinline!An_Org/A!, but not the others: module \code{An_Org/C} has the same vendor, but is located outside of the packaging of module ~\lstinline!An_Org/A.B!, and module \code{Another_Org/D} is completely out of the packaging game.\footnote{The packaging game is too strong for the module \code{D}.} 

The fully qualified names of these modules are as follows: 
\begin{itemize}
\item \lstinline!An_Org/A!
\item \lstinline!An_Org/A.B!
\item \lstinline!An_Org/C!
\item \lstinline!Another_Org/D!
\end{itemize}


Selections ~\lstinline!$p$.$m$!~ from $p$ as well as imports from $p$ work as for objects. It is illegal to have a module with the same fully qualified name (minus the vendor parts) as a class or a trait. 

Top-level definitions outside a packaging are assumed to be injected into the \code{Object} class directly, and therefore visible to each other without qualification. However, as \code{Object} is actually a simple name for the fully qualified name ~\lstinline!global.Aml/Language.Object!, no member is ever defined outside of packaging -- it may only seem to be so: the type of \code{self} pseudo-variable in ``global'' context (outside of any packagings) is ~\lstinline!global.Aml/Language.Object!---a special instance of \code{Object} dedicated to handling ``global'' space---unless the source file is loaded in context of another instance, used with DSLs. This should however be only used in script files, not inside modules, where all code has to be packaged. 










