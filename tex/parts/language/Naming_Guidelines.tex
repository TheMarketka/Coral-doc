%!TEX TS-program = xelatex
%!TEX encoding = UTF-8 Unicode

\chapter{Design Guidelines \& Code Conventions}

\minitoc

This chapter introduces the way programs {\em should} be written and laid out in Aml. 






\section{Introduction}

\subsection{Purpose of Having Code Conventions}

They are important to programmers for many reasons:
\begin{itemize}
  \item Maintenance of a software is a huge part of its lifetime costs. 
  \item Software is usually not maintained for its whole lifetime by its original author. If you think that is not your case, imagine a future world, where an apocalypse happened. People that survived it might need to update your software for the new environment requirements. Or maybe in a different scenario, mankind reaches another terraformed planet and need to update your software for the new planet’s requirements, probably different time and so on. Still not convinced? 
  \item Code conventions improve the readability of the software, allowing engineers to understand new code more quickly and thoroughly. 
  \item If your software is shipped including its source code, you might want to not feel embarrassed about it, and rather ship it clean. 
\end{itemize}






\section{Module Structure \& File Names}
\label{sec:module-structure}

This section lists commonly used module structure and file suffixes and names. 

A module is a directory located in your system at a location unspecified by this specification. In the following sections, it will be referred to as {\em module root}. 

\paragraph{Module directory name}
A directory holding a module must have the same name as the module, contained in a directory of the vendor name. If it is standalone, so that no other modules of the same name share the same parent directory, the vendor part including the leading dot can be omitted.

\example Such directories may look like:
\begin{lstlisting}[deletekeywords={module,native},mathescape=false]
/Aml/Language
/Example/Your_Module
    /source (* or `src` *)
        /aml
            /Some_Submodule 
            /macros
                /some_macro.aml
            /some_source_file.aml
            /module.aml
            /build.aml
            /module.dependencies.aml
            /module.dependencies.lock
        /cpp14 (* or whichever the Aml VM is implemented in *)
            /native_method.hpp
            /native_method.cpp
    /compiled (* or `com` *)
        /aml
            /Some_Submodule
            /Example/Your_Module.amlb
            /Example/Your_Module/native/ (* optimized code *)
            /Example/Your_Module.amlpsi
        /cpp14
            /native_method.a
    /binaries (* or `bin` *)
        /an_executable.a
        /run_your_module (* some executable file 
                            using #! to run an Aml VM *)
    /resources (* or `res` *)
        /logo.png
    /.gitignore
    /.git
\end{lstlisting} 

\paragraph{Sources}
A common name for this directory is \code{src}, or \code{source}. Inside of this directory, all Aml sources may be located, or alternatively, it can be forked into more directories, each for a specific language. 

The directory name consists of a lowercase name of the source language, optionally followed by a platform specifier. A build tool used to create the software may then choose the sources appropriately. 

Aml source files have the suffix \code{.aml}. Readme files are not source files, having their name starting with any-case \code{readme}, followed by e.g. \code{.md} for a Markdown-formatted readme. 

Aml source directory (\code{/src} or \code{/src/aml}) might also contain a build file (or multiple build files, and even along with the language-specific directories in \code{/src} directly -- some may be generated during compilation to provide skeletons for implementation of native methods and functions), which, for Aml, is named \code{build.aml}, and can make use of other files as needed. 

Macros should be placed in a \code{/macros} directory inside of a directory with Aml sources, so e.g. \code{/src/macros} or \code{/src/aml/macros}. Those files may also use other files as needed. Macro definitions may appear outside of that directory too. 

A module file may be placed in a Aml sources directory, and is named \code{module.aml}. This file is important in the fact that it defines the module and vendor name for the module, and the vendor name for all nested submodules. Along with this file, a \code{module.dependencies.aml} and \code{module.dependencies.lock} files may be present, to define the module's dependencies on other modules, where the latter is a generated file, supposed to be checked into your VCS\footnote{Version Control System.}.

This directory may be omitted from a release version of your software (usually useful for proprietary software). 

Submodules can be placed in a directory of the same name as the submodule's inside the directory where Aml sources are. 

\paragraph{Compiled files}
A common name for this directory is \code{com}, or \code{compiled}.

Aml compiled files (bytecode) have the suffix \code{.amlb} for bytecode and \code{.amlpsi} for extracted Program Structure Information. Readme files might be copied into it, if needed. Note that Aml VM will refuse to execute module's entry point if the module is not compiled including all native method implementations. 

Aml native compiled files have suffix defined by the implementation. Could be \code{.dylib}, \code{.so}, \code{.dll}, or whichever the implementation uses. 

Note that files compiled from other languages, if not bound to Aml environment via native method definitions, may be still used with \code{FFI} mechanisms. 

Compiled files do not need to be checked into a VCS, unless they can't be generated from sources (usually applies to files compiled from external sources). 

\paragraph{Binary files}
A common name for this directory is \code{bin}, or very less commonly \code{binaries}. It is intended to contain any necessary binary files that are not to be compiled during build of the software, but rather just included in it as they are. May include executable binary files, or even shell scripts that start up a Aml VM with a specific entry point. 

\paragraph{Resources}
A common name for this directory is \code{res}, or \code{resources}, the latter may be preferred. Such directory is supposed to hold any files that the module needs, may it be images, sounds or whatever. Its organization is up to the module's maintainer. 





\section{File Organization}

A file in Aml is basically a function, but it is a good idea to keep it organized in sections, if it represents a definition rather than a script. Each section should be separated by blank lines and preceded with a documentation comment if needed (usually when the documented section is a part of a public API). 

Files longer than 1989 lines are cumbersome and should be avoided. Aml has a mechanism of open classes that allows programmers to split longer class definitions into multiple files, one defining the basic parts of the class and locations of its other sources and the others should define the implementation, logically separated. 





\subsection{Aml Source Files}

Each Aml source file contains a definition of an implicit function, which is defined by the whole file. Such function may define types, classes, traits, and also do stuff. 

A Aml file whose purpose is to define classes should contain at most one such top-level class (and may include inner classes as needed, indeed). When a private class or interface is associated with a public class, these can be put into the same file. The public class should be the first class defined in the file. 

A typical Aml source file has the following ordering: 
\begin{enumerate}
  \item Module identification, e.g.:
\begin{lstlisting}
module Heroes/Cool_App
\end{lstlisting}

  \item Imports, e.g.:
\begin{lstlisting}
use Them/Cool_Library.{Some_Class as A_Class, _}
\end{lstlisting}

  \item Pragmas for the whole file, e.g.:
\begin{lstlisting}
pragma Profile :Prague
\end{lstlisting}

  \item The class or type definition(s). 
\end{enumerate}






\subsection{Class Definition Organization}

The following list describes the parts of a class definition (or a trait), in the order that they should appear. 

\begin{enumerate}
\item {\bfseries Class documentation} \hfill \\
A class documentation should introduce the class and its purpose to the documentation reader. 

\item {\bfseries Class signature statement} \hfill \\
According to Aml's syntax, this defines the class' name, parents (including traits), type parameters, primary constructor's parameters and annotations for the class itself and its primary constructor. The primary constructor can also define some (or all) of the class' instance variables. The order of each element is defined by the syntax (\sref{sec:class-definitions}). 

\item {\bfseries Class implementation comment} \hfill \\
This comment should contain any class-wide information that is considered essential to understand the implementation and wasn't appropriate for the class documentation. 

\item {\bfseries Type declarations} \hfill \\
Define these early, so that they can be used in the following code. 

\item {\bfseries Superclass instance method invocations} \hfill \\
Use this section to call superclass instance methods. Imagine methods like \code{has_many} from RoR's \code{ActiveRecord}\footnote{RoR stands for Ruby on Rails.}. 

\item {\bfseries Class instance variables} \hfill \\
These are variables whose names are prefixed with a double at-sign ``\code{@@}''. Their order should be in decreasing visibility (\sref{sec:modifiers}). Alternatively, if the class object has a separate definition using \code{object} keyword, that should be following the class definition in the same file, preserving the order defined in this list. This also applies to class instance methods. 

\item {\bfseries Class instance methods} \hfill \\
These methods are defined for the class object and may appear in a separate object definition. Their order should be in decreasing visibility (\sref{sec:modifiers}). Furthermore, factory methods should precede all other methods. 

\item {\bfseries Instance variables} \hfill \\
These are variables whose names are prefixed with a single at-sign ``\code{@}''. Their order should be in decreasing visibility (\sref{sec:modifiers}).

\item {\bfseries Auxiliary constructors} \hfill \\
Their order should be in decreasing visibility (\sref{sec:modifiers}).

\item {\bfseries Methods and other members} \hfill \\
These members should be grouped by functionality rather than by scope or accessibility. This section includes everything else, including method definitions, message declarations, inner classes. 
\end{enumerate}





\section{Indentation}

Tabs with two space displayed width are recommended to be used as the unit of indentation. The displayed width may be customized. If further indentation is required beyond indentation of the scope (e.g. to align type names in successive variable definitions), spaces has to be used for the extra indent (so-called ``smart tabs''), but never for the base indentation. 





\subsection{Line Length}

Avoid lines longer than 80 characters and try not to exceed 100 characters per line top. Use line breaks to achieve shorter lines. 






\subsection{Line Wrapping}

When an expression will not fit into a single line, break it according to the following general principles. Note though that Aml is also possibly used in a REPL environment, therefore it's syntax uses line breaks as significant whitespace in many places. This can however be prevented by means of preventing an expression end -- usually by wrapping an expression that is supposed to be continued on the next line in parentheses. 

\begin{itemize}
\item Break after a comma. 
\item Break before an operator. 
\item Prefer higher-level breaks to lower-level breaks. 
\item Align the new line with the beginning of the expression at the same level on the previous line. Spaces have to be used for this. 
\item If the previous rules lead to confusing code or to code that's squished up against the right margin, just indent two tabs instead. 
\item If the main expression on the line is an assignment or a return statement, don't forget to wrap the assigned expression in parentheses, if it's to be wrapped on the next line. Otherwise, a compile-time error would be raised, as the following operator applications or function applications could not be chained. 
\item Don't break before a closing parenthesis of a function application.  
\end{itemize}

\example Here are some examples of breaking function applications: 

\begin{lstlisting}
(* preferable *)
function_1 (long_expression_1, long_expression_2, long_expression_3,
            long_expression_4, long_expression_5)
val a := function_1 (long_expression_1,
                     function_2 (long_expression_2,
                                 long_expression_3))
(* acceptable *)
function_1 (long_expression_1, long_expression_2, 
    long_expression_3, long_expression_4, 
    long_expression_5)
val a := function_1 (
    long_expression_1,
    function_2 (
        long_expression_2,
        long_expression_3))
\end{lstlisting}

\example Following are examples of breaking an expression with operators. The first is preferred, since the break occurs outside the parenthesized expression, which is at a higher level. 

\begin{lstlisting}
(* prefer *)
long_name_1 := long_name_2 * (long_name_3 + long_name_4 - long_name_5)
               + 4 * long_name_6
(* avoid *)
long_name_1 := long_name_2 * (long_name_3 + long_name_4 
                              - long_name_5) + 4 * long_name_6
\end{lstlisting}

\example Following are examples of indenting method definitions. The first is the conventional case, the second would shift the second and third line to the far right if it used conventional indentation, so instead it indents two tabs. Also, if a visibility modifier is causing the line to be too long, consider using it as a pragma before the method definition instead (but remember that definitions following it will have the same visibility). 

\begin{lstlisting}
(* conventional indentation *)
def some_method (an_arg: Integer, another_arg: Object, 
                 yet_another: String,
                 and_still_another: Object): Unit := {
  $\ldots$
}

(* indent two tabs to avoid very deep indents *)
private method self.very_long_method_name (an_arg: Integer,
    another_arg: Object, yet_another_arg: String, 
    and_still_another: Object): Unit := {
  $\ldots$
}
\end{lstlisting}

\example Line wrapping conditional statements should generally use the two tabs rule, since conventional one tab indentation makes seeing the body difficult. For example: 

\begin{lstlisting}
(* don't use this indentation *)
if ((condition_1 && condition_2)
  || (condition_3 && condition_4)
  || not (condition_5 && condition_6))
  do_something_about_it (* this line is easy to miss *)
end

(* use this indentation instead (notice the extra spaces) *)
if (       (condition_1 && condition_2)
    ||     (condition_3 && condition_4)
    || not (condition_5 && condition_6))
  do_something_about_it
end

(* or use this *)
if (     (condition_1 && condition_2)
  ||     (condition_3 && condition_4)
  || not (condition_5 && condition_6))
then
  do_something_about_it (* this line is now harder to miss *)
end
\end{lstlisting}

\example Aml does not have any ternary operators, but the following is an equivalent expression:

\begin{lstlisting}
alpha := if a_long_boolean_expression
           beta
         else
           gamma
         end

(* or alternatively *)
alpha := if a_long_boolean_expression
  beta
else
  gamma
end
\end{lstlisting}





\section{Comments}

Aml programs can have two kinds of comments: implementation comments and documentation comments. Implementation comments are delimited by block \code{(* $\ \ldots$ *)}. Documentation comments are delimited by \code{(*! $\ \ldots$ *)} or alternatively \code{(** $\ \ldots$ *)}\footnote{As a convenience for those used to this syntax from other languages. Aml toolchain prefers the first version.}. Documentation comments can be extracted by Aml toolchains.

Implementation comments are meant for commenting-out code (to temporarily disable it) or for comments about the particular implementation. Documentation comments are meant to describe the specification of the code, independently on the actual implementation. 

Comments should be used to provide additional information that is not readily available in the code itself, or can't be, e.g. information how the corresponding module is build. 

Discussion of non-trivial or non-obvious design decisions is appropriate, but should not duplicate information that is already present (and clearly visible) in the code. Such redundant comments are likely to get out of date and deprecated. 

\paragraph{Note}
If too many comments are necessary to be included in the code, it might be a sign of a poor quality of the code -- consider rewriting it to make it clearer and get rid of the extra comments. 

Comments should not be enclosed in any decorative boxes made of asterisks or whatever else. 





\section{Declarations}






\subsection{One Per Line}

One declaration per line is recommended, since it encourages commenting, unless the declarations are a part of a pattern matching extraction. 

Multiple declaration expressions on the same line using a semicolon to separate them should be avoided, and should definitely not mix variable and function declarations. 

\example The following shows preferred variable declarations and declarations to be avoided:
\begin{lstlisting}
(* preferred *)
val level: Integer
val size: Integer

(* avoid *)
val level, size: Integer
val level: Integer; size: Integer
\end{lstlisting}

\example Another acceptable alternative is to align the type names with spaces: 
\begin{lstlisting}
val level:         Integer
val size:          Integer
val current_entry: Object
\end{lstlisting}





\subsection{Placement}

Preferable is to put declarations at the beginning of scopes (blocks). Variable declarations in the middle of code should be avoided. The only exception is in expressions like generators. 

To optimize variable creation, variables do not have to be defined before first used -- Aml provides the \code{lazy} modifier for variables, which takes care of that. 





\subsection{Initialization}

Try to initialize every variable as soon as possible, or use the \code{lazy} modifier on it. 






\section{Class Definitions}

When writing Aml classes, the following formatting rules should be followed: 
\begin{itemize}
  \item A space between a method name and the parenthesis of its first parameter list, if any. 
  \item Open brace ``\lstinline!{!'' appears at the end of the same line as the declaration statement. 
  \item Open keyword \code{begin} appears on the next line to the declaration statement. 
  \item Closing brace ``\lstinline!}!'' or close keyword \code{end} appears on a line alone (or followed by the keyword appropriate for the closed term). The only exception is when the close brace or close keyword should end the declaration immediately (usually in method declarations or definitions of methods that have an empty body, thus returning a \code{Unit} value). 
  \item The \code{declare} and \code{begin} keywords should appear alone on their lines. 
  \item Members except instance variables and class instance variables are separated by a blank line. 
\end{itemize}





\section{Statements}






\subsection{Simple Statements}

Each line should contain at most one statement. Avoid grouping statements on one line, unless there is an obvious reason to do so. 





\subsection{Compound Statements}

Compounds statements are those enclosed in a block. 
\begin{itemize}
\item The enclosed statements should be indented one more level than the compound statement. 
\item The opening brace or keyword should be at the end of the line that begins the compound statement. The closing brace or keyword should begin a line and be indented to the beginning of the compound statement, unless that indent would be too deep. 
\end{itemize}





\subsection{Return Statements}

A \code{return} statement with a value should not use parentheses unless required by the syntax.

\example For instance: 
\begin{lstlisting}
return

return my_disk.size

return (builder
    .set_title "hello"
    .set_message "world"
    .build)
\end{lstlisting}














