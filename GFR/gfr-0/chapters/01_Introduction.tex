%!TEX TS-program = xelatex
%!TEX encoding = UTF-8 Unicode

\chapter{Introduction}

\minitoc

\newpage

% TBD: add an introduction





\section{Notational Conventions}

All lexical and grammar syntax definitions use a customized form of EBNF. 

\begin{itemize}
  \item \lstinline![element]! \newline
    Option, one or zero occurrences of \code{element}. 
  \item \lstinline!{element}! \newline
    Optional repetition, zero, one or more occurrences of \code{element}.
  \item \lstinline!{element}+! \newline
    Repetition, one or more occurrences of \code{element}.
  \item \lstinline!? text ?! \newline
    Special sequence, textual description of an element. The following lines could contain additional data, if so described.
  \item \lstinline!element1 | element2! \newline
    Alternation, either \code{element1} or \code{element2} applies, but not both. 
  \item \lstinline!(element)! \newline
    Grouping. 
  \item \lstinline!element1, element2! \newline
    Concatenation, used only for exceptions to enlist elements rather than sequence them. 
  \item \lstinline!element1 element2! \newline
    Sequence of elements.
  \item \lstinline!'data'! \newline
    Terminal string, literal appearance of the given Unicode characters. 
  \item \lstinline!| $\cdots$ |! \newline
    Signals a range from the previous alternative to the following alternative. 
  \item \lstinline!element - except! \newline
    Exception, an anonymous element that contains all of \code{element}, except for elements defined by \code{except} (which can be a grouping). 
  \item \lstinline!category ::= element! \newline
    Used for definitions; signals that \code{category} is made of \code{element}, as in EBNF. The same \code{category} could appear on the left side multiple times for \code{element$_1$} \ldots\ \code{element$_n$}, in that case, it is taken that \lstinline!category ::= element$_1$ | $\ldots$ | element$_n$!.
  \item \lstinline!(* comment *)! \newline
    An additional information about the given syntax element, could contain additional restrictions described textually. 
\end{itemize}







