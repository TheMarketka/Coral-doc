%!TEX TS-program = xelatex
%!TEX encoding = UTF-8 Unicode

\chapter{Type Definitions}

\minitoc

\newpage

\section{Aspects}
\label{sec:aspects}

\syntax\begin{lstlisting}
Tmpl_Def        ::= 'aspect' [Aspect_Clause] Aspect_Def
Aspect_Def      ::= id ['extends' Class_Parents] Aspect_Tmpl_Env
Aspect_Tmpl_Env ::= '{' Aspect_Body '}'
                  | 'begin' Aspect_Body 'end' ['aspect']
Aspect_Body     ::= Template_Stat 
                  | Aspect_Stat
Aspect_Stat     ::= Advice_Def
                  | Pointcut_Def
Pointcut_Def    ::= 'pointcut' id
                    ['(' [id [':' Type]] {',' [id [':' Type]]} ')']
                    'is' Pointcut_Type 'end' ['pointcut']
Pointcut_Type   ::= 'invoke' (id | regular_expression_literal)
                  | 'get' (id | regular_expression_literal)
                  | 'set' (id | regular_expression_literal)
                  | 'handler' Type
                  | 'returning' [id [':' Type] | Type]
                  | 'throwing' [id [':' Type] | Type]
                  | 'raising' [id [':' Type] | Type]
                  | 'yielding' [id [':' Type] | Type]
                  | 'advice-execution' 
                    [(id | regular_expression_literal)]
                  | 'self' (id | Type)
                  | 'target' (id | Type)
                  | 'arguments' '(' id [':' Type] | Type 
                    {',' (id [':' Type] | Type)} ')'
                  | 'if' (Infix_Expr | '(' Infix_Expr ')')
                  | 'not' Pointcut_Ref
                  | Pointcut_Ref 'and' Pointcut_Ref
                  | Pointcut_Ref 'or' Pointcut_Ref
                  | '(' Pointcut_Ref ')'
                  | 'retain' [id [':' Type] | Type]
                  | 'release' [id [':' Type] | Type]
Pointcut_Ref    ::= id ['(' id {',' id} ')'] | Pointcut_Type
Advice_Def      ::= 'advice' 
                    Advice_Spec (Pointcut_Ref | Pointcut_Type)
                    Block_Expr
Advice_Spec     ::= 'before' 
                  | 'after' [('returning' | 'throwing' | 'raising'
                            | 'yielding') [id [':' Type] | Type]]
                  | 'around'
Aspect_Clause   ::= 'per-self' [Pointcut_Ref] 
                  | 'per-target' [Pointcut_Ref] 
\end{lstlisting}

Only for \code{Block_Expr} inside of \code{Advice_Def}:
\begin{lstlisting}
Simple_Expr1    ::= 'joinpoint' ['.' Selection]
\end{lstlisting}

% https://eclipse.org/aspectj/doc/released/progguide/semantics-pointcuts.html
% https://eclipse.org/aspectj/doc/released/progguide/quick.html (better)
% important term: published join point context data (args, target, caller etc.)
% also important: http://en.wikipedia.org/wiki/Cross-cutting_concern#Examples - see memory management (possible join points: allocation, deallocation, retain, release)






\section{Union Cases}
\label{sec:unions}

\syntax\begin{lstlisting}
Const_Type_Def ::= id [Type_Param_Clause] 'is' Case_Union
Case_Union     ::= 'case' 'union' 'of' 
                   id ['of' Type] {semi id ['of' Type]}
                   [semi Template_Body]
\end{lstlisting}

Union types represent multiple types, possibly unrelated. Union types are abstract by nature and can not be instantiated, only the types that they contain may, if these are instantiable. For type safety, bindings of union types should be matched for the actual type prior to usage. 

Unions are indeed virtually ``tagged'' with the actual type that they represent at the runtime moment, although when it comes to overloading resolution, the union type is used, as it is the expected type. 

Types contained in a union type of form \code{case union $t$} are given a case name, with the syntax form of ~\lstinline!$a$ of $T$!, where $a$ is the case name. The case name can be used as extractor in pattern matching:
\begin{lstlisting}
type Message is case union of 
  Result of String
  Request of Integer * String
  def name := match self 
    when Result(nm) then nm
    when Request(_, nm) then nm
  end match
end type
\end{lstlisting}
Such a union type may contain the optional template body, and is called {\em case union type}. A case union type is a simplified form for a series of case classes and case objects. 







\section{Enums}
\label{sec:enums}

\syntax\begin{lstlisting}
Const_Type_Def     ::= id [Type_Param_Clause] [Class_Param_Clauses]
                       ['extends' Class_Parents] 
                       'is' ['bitfield' | 'case'] 
                       'enum' '(' Enum_Field {semi Enum_Field} ')'
                       [Enum_Body]
Enum_Field         ::= id [':=' scalar_literal]
                     | id {'(' [Exprs] ')'}
                     | id Class_Param_Clauses [Case_Enum_Body]
Enum_Body          ::= Template_Body
Case_Enum_Body     ::= '{' [Template_Body] '}'
                     | 'begin' Template_Body 'end'
\end{lstlisting}

Enums (short for Enumerations) are types that contain constants. Bitfield enums may be combined to still produce a single enum value. Every enum constant is a singleton instance of the enum class. 

An enum constant can be enhanced with custom parameters, which are then passed to the appropriate custom constructor. Enum definitions should only appear in four forms: 
\begin{itemize}
  \item Initialized with a scalar literal. Each enum constant then have a method \code{value}, which corresponds to the passed literal. Each literal passed has to be of the same type and numeric literals get an auto-incremented value for every following enum constant definition that omits its explicit value. At least the first enum constant has to be initialized with a scalar literal. The comparison operators (e.g. ``\code{<}'') are automatically added and their implementations reflect the order in which the enum constants appear. The signature of the enum must extend a scalar type, and is the only one that allows for the \code{bitfield} keyword to be used with it.
  \item Initialized with nothing, then each enum constant is by itself the ordering, no implicit literal value is added. The comparison operators are automatically added and their implementations reflect the order in which the enum constants appear. 
  \item Initialized with custom expressions. Each argument expression has to have a corresponding constructor defined in the enum template body or use the primary constructor. It is highly recommended for every enum constant to be immutable, but it is not mandatory. This form can't extend any parent class and has the same implicit ordering as the previous form. 
  \item Initialized like a case class. Each enum item defines its own parameters and those then behave like case classes. The \code{case} keyword may or may not be used to enforce this definition form: if it is present, then this form is required for each item, meaning that the enum can not extend a scalar type. If primary constructor is defined, then each item has to use it. If a parent class is defined, then the constructors must make sure that the parent's designated constructor is invoked. The generated \code{apply} method is memoized and referentially transparent, thus lowering the possible number of instances of each case class. 
\end{itemize}




\section{Record Types}
\label{sec:record-types}

\syntax\begin{lstlisting}
Const_Type_Def    ::= Record_Name Is ['abstract'] 'record'
                      (Record1 | Record2)
Record1           ::= Record_Components [Record_Extensions]
Record2           ::= '{' Record_Components '}' [Record_Extensions]
Record_Name       ::= id [Type_Param_Clause]
Record_Components ::= Record_Component {semi Record_Component}
Record_Component  ::= ['val' | 'var'] id ':' (Type | id)
                    | 'case' id Record_Cases
                      {'when' id 
                      ('then' | nl) 
                      Record_Components} 'end' ['case']
Record_Cases      ::= semi Record_Case {semi Record_Case} 'end' ['case']
                    | '{' Record_Case {semi Record_Case} '}'
Record_Case       ::= 'when' ['.'] Stable_Id ['(' [Extractions] ')']
                      ('then' | semi) Record_Components
Record_Extensions ::= 'with' Record_Extension {'and' Record_Extension}
Record_Extension  ::= Tmpl_Member
                    | Tmpl_Ifc_Impl
                    | Tmpl_Ifc_Dcl
\end{lstlisting}





\section{Range, Floating \& Fixed Point Subtype Definitions}
\label{sec:fl-fi-subtypes}

\syntax\begin{lstlisting}
Type_Def        ::= FP_Type_Def 
                  | FP_Subtype_Def 
                  | FP_Range_Def
FP_Type_Def     ::= id [Type_Param_Clause] (':=' | 'is')
                    (FiP_Type_Def | FlP_Type_Def) 
FP_Subtype_Def  ::= id [Type_Param_Clause] (':=' | 'is') 
                    Type
                    (FiP_Subtype_Def | FlP_Subtype_Def) 
FP_Range_Def    ::= id [Type_Param_Clause] (':=' | 'is') 
                    FP_Range
FlP_Type_Def    ::= FP_Digits [FP_Range]
FlP_Subtype_Def ::= FP_Digits [FP_Range]
                  | FP_Range
FiP_Type_Def    ::= FP_Delta [FP_Range] 
                  | FP_Delta FP_Digits [FP_Range]
FiP_Subtype_Def ::= FP_Delta [FP_Digits] [FP_Range]
                  | FP_Digits [FP_Range]
                  | FP_Range
FP_Digits       ::= 'digits' Infix_Expr
FP_Delta        ::= 'delta' Infix_Expr
FP_Range        ::= 'range' Infix_Expr 
                    ('..' | '...' | '..<') Infix_Expr
                  | 'range' Type
\end{lstlisting}

The described syntaxes are for definitions of 4 special types of values:
\begin{enumerate}
  \item Range subtypes.
  \item Floating point types. 
  \item Ordinary fixed point types. 
  \item Decimal fixed point types. 
\end{enumerate}

\paragraph{Range subtypes}
A range subtype (the \code{FP_Range_Def} syntax category) is a type defined by a lower and upper bounds. The expected type of both bounds is \code{Magnitude}. Such a range type may be used in combination with the following subtypes to constrain them, or standalone as a regular range value. The lower bound must be lower than or equal to the upper bound. The lower bound may be negative infinity, the upper bound may be positive infinity. The range subtype itself is a subtype of \code{Magnitude}, or more precisely, a subtype of the least upper bound of types of the bounds, selecting a range of its values. For floating and fixed point types, it has to be a range of \code{Real} values. 

\paragraph{Floating point types} ~\\
A floating point type (the \code{FlP_Type_Def} syntax category) is a way to define an appropriate representation of a floating point number, based on the required accuracy instead. 
\begin{itemize}
  \item[] The {\em requested decimal precision}, which is the minimum number of significant decimal digits required for the floating point type, is specified by the value of the expression given after the keyword \code{digits}. Such expression is expected to be of \code{Integer} type. 
  \item[] The bounds of the range specification are expected to be \code{Real} type; the types do not need to be the same.\footnote{E.g., one bound can be an integer, the other a real number.}
  \item[] The requested decimal precision shall be positive and not greater than an implementation-defined precision limit in \code{Number.Max_Base_Digits}. If the range specification is omitted, the requested decimal precision shall be not greater than \code{Number.Max_Digits}. 
  \item[] A floating point type definition is illegal if the implementation does not support a floating point type that satisfies the requested decimal precision and range. 
  \item[] A subtype of a floating point type is compatible to the parent type if the digits of the subtype are not greater than the digits of the parent type, and its range fits to the range of the parent type. 
\end{itemize}

\example Examples of floating point types and subtypes:
\begin{lstlisting}
type Coefficient is digits 10 range -1.0 ..< 1.0 
end type

type Mass is digits 7 range 0.0 ..< 1.0e+35 
end type

(* a subtype with a smaller range *)
type Probability is Real range 0.0 ... 1.0
end type
\end{lstlisting}

\paragraph{Fixed point types}
An ordinary fixed point type (the first branch of the \code{FiP_Type_Def} syntax category) is a way to define a decimal type, based on the given delta. 
A decimal fixed point type (the second branch of the \code{FiP_Type_Def} syntax category) is a way to define a decimal type, based on the given delta and number of needed digits.  
\begin{itemize}
  \item[] The error bound of a fixed point type is specified as an absolute value, called the {\em delta} of the fixed point type. 
  \item[] For a type defined by the fixed point type definition, the delta of the type is specified by the value of the expression given after the keyword \code{delta}; this expression is expected to be of a \code{Real} type. For a type defined by the decimal fixed point definition, the number of significant decimal digits is specified by the expression given after the keyword \code{digits}; this expression is expected to be of \code{Integer} type. 
  \item[] The expressions given after the reserved keywords \code{delta} and \code{digits} shall result in positive values. 
  \item[] The set of values of a fixed point type comprise the integral multiples of a number called the {\em small} of the type. 
  \item[] For ordinary fixed point type, the small is an implementation-defined power of 2 not greater than the delta, unless annotation \code{@[fixed_point_small $s$]} is applied to the type, defining the small to be $s$, where $s$ is not greater than the delta. 
  \item[] For decimal fixed point type, the small equals the delta; the delta shall be a power of 10. If a range specification is given, both bounds of the range shall be in the range $-(10^{\digits}-1)*\fpdelta$ \code{...} $(10^{\digits}-1)*\fpdelta$. 
  \item[] A fixed point type definition is illegal if the implementation does not support a fixed point type with the given small and specified range or digits. 
  \item[] An ordinary fixed point type definition defines an ordinary fixed point type, whose base range includes at least all multiples of the small that are between the bounds defined by the range specification, if it is given, or between negative infinity to positive infinity\footnote{Using the \code{Decimal} types.}, if the range specification is not given.
  \item[] A decimal fixed point type definition defined a decimal fixed point type, whose base range includes at least the range $-(10^{\digits}-1)*\fpdelta$ \code{...} $(10^{\digits}-1)*\fpdelta$. 
  \item[] If a decimal fixed point type definition does is not given a range specification, then an implicit range $-(10^{\digits}-1)*\fpdelta$ \code{...} $(10^{\digits}-1)*\fpdelta$ is specified for it. 
  \item[] A subtype of a decimal fixed point type is compatible to the parent type if the digits of the subtype are not greater than the digits of the parent type, and its range fits to the range of the parent type. 
\end{itemize}

\example Examples of fixed point types and subtypes:
\begin{lstlisting}
type Volt is delta 0.125 range 0.0 ..< 255.0 
end type

type Fraction is delta Number.Fine_Delta range -1.0 ..< 1.0 
end type

type Money is delta 0.01 digits 15 
end type

type Salary is Money digits 10 
end type
\end{lstlisting}





