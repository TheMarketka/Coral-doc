%!TEX TS-program = xelatex
%!TEX encoding = UTF-8 Unicode

\chapter{Annotations, Pragmas \& Macros}
\label{sec:annotations}
\label{sec:pragmas}
\label{sec:macros}

% TBD: add annotations that define named parameter types, captured named parameters, repeated parameter and captured block parameter, so that traits lik Function_n may use it in its type parameters to apply the parameter extras to the apply method

\syntax\begin{lstlisting}
Annotation     ::= '@[' Simple_Type [NB_Arg_Exprs] ']'
NB_Arg_Exprs   ::= Parens_Args {Parens_Args}
                 | Poetry_Args
Annotation_Def ::= 'annotation' Class_Def
Expr           ::= Pragma
Pragma         ::= 'pragma' Simple_Type [NB_Arg_Exprs]
Def            ::= 'def' 'macro' Fun_Def 'end' ['def']
                 | 'def' 'macro' Fun_Alt_Def
\end{lstlisting}

Annotations, pragmas \& macros are a way to provide metadata to both the compiler and runtime of Gear, possibly affecting the resulting bytecode and abstract syntax trees. 





\section{Annotations}

Annotations are classes that must conform to \code{Annotation}, and which can thus be applied in annotated expressions and annotated definitions or types. 

Annotations always appear before element that they annotate, and if multiple annotations are applied, their order is preserved in respect of reflection, although the actual order may or may not matter. 

% TBD: add reference of annotations as they become clear and well-defined
% as now there are just a few mentioned throughout the CLS, but it's too few to properly categorize them





\section{Pragmas}
\label{sec:pragmas}

Pragmas are basically annotations that are applied in its scope from that point on and in any nested scopes, not binding to just a single expression, definition or a type. Some annotations can only be applied as a pragma. 






\section{Macros}
\label{sec:macros}

Macros are a way to directly manipulate with abstract syntax trees. Unlike in languages such as C, macros in Gear are written using the same language. The only essential restriction here is that while compiling a Gear module or another source file, every macro that is applied in it must be pre-compiled, e.g. available from a separate compilation phase, and applied in the same compilation phase, not runtime. The only way to apply macros in runtime is by ad-hoc compilation. 

Macro authors are encouraged to use syntactic forms (\sref{sec:syntactic-forms}) to manipulate and generate abstract syntax trees. 

A macro is defined as a regular function, but it's body is required to pass invocation to a method that implements the macro body, where every parameter type $T$ is replaced with ~\lstinline!c.Expr[$T$]!, and where \code{c} is the first parameter of type \code{Context}. The macro implementation has exactly two parameter lists: 
\begin{enumerate}
  \item A parameter list with exactly one parameter of type either
    \begin{itemize}
      \item \code{Lang~[gear].Reflection.Macros.Whitebox.Context} or 
      \item \code{Lang~[gear].Reflection.Macros.Blackbox.Context}.
    \end{itemize} 
  \item A parameter list with the parameters of the macro definitions, with the described parameter type translation applied. 
\end{enumerate}

\example The following code is an example implementation of a simple \code{assert} macro. 
\begin{lstlisting}[deletekeywords={message}]
;; import a blackbox context
use Lang~[gear].Reflection.Macros.Blackbox.Context

;; define the macro
def macro assert (condition: Boolean, message: String_Like): Unit := 
    Asserts.assert_impl

;; implement the macro
object Asserts {
  def assert_impl 
      (c: Context)
      (cond: c.Expr[Boolean], msg: c.Expr[String_Like]): Unit :=
    `(
      unless #{cond}
        raise #{msg}
      else
        ()
      end
    )
}
\end{lstlisting}

Macros can be applied as an intermediate step by IDEs, so that their resulting transformations could be viewed prior to proceeding with compilation. 





\subsection{Whitebox \& Blackbox Macros}
\label{sec:whitebox-blackbox-macros}

{\em Blackbox macros} are such macros that exactly follow their type signature, including the result type, and therefore can be treated as blackboxes. Their implementations are irrelevant to understanding their behavior. On the otherhand, {\em whitebox macros} do not necessarily follow their type signature, which they have, but only as an approximation, which may or may not be precise. Therefore, whitebox macros may be used to create e.g. type providers, functional dependency materialization or extractor macros. 

Blackbox macros have the following restrictions applied to them:
\begin{enumerate}
  \item As an application of a blackbox macro expands into a tree $x$, the expansion is wrapped in a typed expression ~\lstinline!($x$ as $T$)!, where $T$ is the declared result type of the blackbox macro with type arguments and path dependencies applied in consistency with the particular macro application being expanded. This invalidates blackbox macros as a possible implementation of type providers. 
  \item When an application of a blackbox macro still has undetermined type parameters even after type inference, these type parameters are forcedly inferred, in the same manner as type inference works for normal methods. This invalidates blackbox macros from creating functional dependency materialization. On the contrary, whitebox macros defer type inference of undetermined type parameters (\code{Undefined}) until the macro application is expanded. 
  \item When an application of a blackbox macro is used as an implicit candidate, no expansion is performed until the particular macro is finally selected as the result of the search for an implicit. 
  \item It is an error if a blackbox macro is used as an extractor in pattern match. 
\end{enumerate}






\subsection{Macro Annotations}
\label{sec:macro-annotations}

Macro annotations are a combination of macros and annotations -- such that a macro annotation is basically an annotation that defines a method \code{apply_macro} with a single macro definition, where the definition has exactly one parameter list with exactly one parameter of a reflection type that the macro annotation can be applied to. The same annotation may also define the implementation method of the macro, but that is not required. The macro definition has to return a single value, or multiple values via a tuple. The single value may be \code{Unit}, therefore effectively discarding the annotated expression or definition. 

Macro annotations are the most suitable way to create type providers. 

\paragraph{Note}
With macro annotations, the order of appearance of the annotations is inherently important, since the reflection type passed to the macro's implementation could be different with each different order of macro annotation applications. 






