%!TEX TS-program = xelatex
%!TEX encoding = UTF-8 Unicode

\chapter{Memory Models}
\label{sec:memory-models}

\minitoc

Gear offers to its proper implementations these memory models:
\begin{itemize}
  \item Automatic Reference Counting (\sref{sec:arc-memory-model})
  \item Garbage Collection (\sref{sec:gc-memory-model})
\end{itemize}

Each memory model has its pros and cons, discussed further in the following sections. 






\section{Automatic Reference Counting}
\label{sec:arc-memory-model}

A Gear implementation with {\em automatic reference counting} tracks reference counts (strong reference count, weak reference count) of all heap-allocated values. Whenever strong reference count drops to 0, the referenced value is destructed (by invoking destructors on it and maybe destructing other values it itself referenced), and its memory is reclaimed by the Gear VM. Therefore, specific time of value's destruction may be determined, and the runtime does not need to ``pause the world'' to perform any memory clean-up tasks, which could cause interrupts in the program's behaviour. 

Such implementation does not need to support soft and phantom references. 









\section{Garbage Collection}
\label{sec:gc-memory-model}

A Gear implementation with {\em garbage collection} does not destruct values that are no longer referred to immediately, but does so at unspecified times in the future, by ``pausing the world'' to detect those values it can safely destruct. 

Such implementation can and should support soft and phantom references. 













