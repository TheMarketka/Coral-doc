%!TEX TS-program = xelatex
%!TEX encoding = UTF-8 Unicode

\chapter{Units of Measure}
\label{sec:units-of-measure}

\minitoc

\newpage

\section{Units of Measure}

\syntax\begin{lstlisting}
Const_Type_Def ::= Unit_Name 'is' ['abstract'] 'measure' 
                   [semi Unit_Convs {semi Unit_Convs}]
Unit_Name      ::= id - Unit_Op ['extends' id - Unit_Op]
Unit_Convs     ::= Unit_Name ':=' Unit_Conv
Unit_Conv      ::= '(' Unit_Conv ')'
                 | Unit_Elem [Unit_Op Unit_Elem]
                 | Unit_Conv Unit_Op Unit_Conv
Unit_Elem      ::= number_literal | id |
Unit_Op        ::= '*' | '/' | '^'
\end{lstlisting}

Numbers in Gear can have associated units of measure, which are typically used to indicate length, volume, mass, distance and so on. By using quantities with units, the runtime is allowed to verify that arithmetic relationships have the correct units, which helps prevent programming errors. 

\example The following defines the measure \code{cm} (centimeter).
\begin{lstlisting}
type cm is measure
end type
\end{lstlisting}

\example The following defines the measure \code{ml} (milliliter) as a cubic centimeter (\lstinline!cm ^ 3!).
\begin{lstlisting}
type ml is measure
  ml := cm ^ 3
end type
\end{lstlisting}

\example The following shows possible usage of abstract units of measure. 
\begin{lstlisting}
type distance is abstract measure
end type

type m extends distance is measure
  m := km / 1000
end type

type km extends distance is measure
  km := m * 1000
end type

type mi extends distance is measure
  mi := km * 0.621
  mi := (m * 1000) * 0.621 (* this can be inferred! *)
end type
\end{lstlisting}
This enables types aggregated with units of measure require a number tagged with any distance unit of measure and still work with correct units. 

Every unit of measure is defined in the same scope as any other type would be, but the application of units of measure to numbers or {\em aggregated unit types} require to import units of measure by name into the scope where a unit of measure from a different unrelated module would be used. 

\paragraph{\em Types Aggregated with Units of Measure}
In addition to type parameters of each type, every type may be parameterized with a units of measure aggregation. These parameters are each put into angle brackets, and the only available bound is an upper bound for abstract unit of measure type. 

\syntax\begin{lstlisting}[mathescape=false]
Variant_Type_Param ::= '<' (id | '_') ['<:' id] '>'
\end{lstlisting}

Names of unit of measure parameters must not clash with names of type parameters or other unit of measure parameters, otherwise it is a compile-time error. 

\paragraph{\em Persistence of Units of Measure}
There is a huge difference between the way F\# handles units of measure and Gear's way. In F\#, the unit of measure information is lost after compilation, but persists in Gear in runtime, since verification of units of measure is deferred also to runtime, as it is limited during compilation. This also means that the information may be accessed in runtime, e.g. using it to print the unit information on screen.

